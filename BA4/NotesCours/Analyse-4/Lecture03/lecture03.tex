% !TeX program = lualatex
% Using VimTeX, you need to reload the plugin (\lx) after having saved the document in order to use LuaLaTeX (thanks to the line above)

\documentclass[a4paper]{article}

% Expanded on 2023-03-09 at 15:15:30.

\usepackage{../../style}

\title{Analyse 4}
\author{Joachim Favre}
\date{Jeudi 09 mars 2023}

\begin{document}
\maketitle

\lecture{3}{2023-03-09}{Des courbes simples sur un plan complexe}{
\begin{itemize}[left=0pt]
    \item Définition de courbe régulière, courbe simple, courbe fermée et courbe régulière par morceaux.
    \item Définition de l'intégrale complexe.
    \item Explication du théorème de Cauchy.
    \item Explication de la formule intégrale de Cauchy.
\end{itemize}

}

\subsection{Intégration complexe}
\parag{Définition: Courbe régulière}{
    L'ensemble $\Gamma \subset \mathbb{C}$ est une \important{courbe régulière} s'il existe $\left[a, b\right] \subset \mathbb{R}$, et une fonction $\gamma$, appelée une \important{paramétrisation} de $\Gamma$:
    \[\begin{split}
    \gamma: \left[a, b\right] &\longmapsto \Gamma \subset \mathbb{C} \\
    t &\longmapsto \gamma\left(t\right) = \alpha\left(t\right) + i\beta\left(t\right)
    \end{split}\]
    
    Cette paramétrisation doit suivre les propriétés suivantes:
    \begin{enumerate}
        \item $\Gamma = \left\{\gamma\left(t\right) \suchthat t \in \left[a, b\right]\right\}$.
        \item $\gamma \in C^{1}\left(\left]a, b\right[, \mathbb{C}\right)$.
        \item $\gamma'\left(t\right) = \alpha'\left(t\right) + i \beta'\left(t\right) \neq 0$ pour tout $t \in \left]a, b\right[$.
    \end{enumerate}
}

\parag{Définition: Courbe régulière simple}{
    Une courbe régulière $\Gamma$ de paramétrisation $\gamma: \left[a, b\right] \mapsto \Gamma$ est \important{simple} si $\gamma$ est injective sur $\left]a, b\right[ $. 

    En d'autres mots, on veut que, pour tout $t_1, t_2 \in \left]a, b\right[$ tels que $t_1 \neq t_2$:
    \[\gamma\left(t_1\right) \neq \gamma\left(t_2\right)\]

    \subparag{Intuition}{
        Intuitivement, cela veut simplement dire que la courbe ne s'intersecte pas.
    }

    \subparag{Remarque}{
        Nous n'étudierons que les courbes régulières simples dans ce cours.
    }
}

\parag{Définition: Courbe régulière fermée}{
    Une courbe régulière $\Gamma$ de paramétrisation $\gamma: \left[a, b\right] \mapsto \Gamma$ est \important{fermée} si: 
    \[\gamma\left(a\right) = \gamma\left(b\right)\]
    
    \subparag{Remarque}{
        Une courbe peut être régulière simple et fermée. En effet, l'injectivité n'est demandée que sur $\left]a, b\right[ $.
    }
}

\parag{Définition: Intérieur}{
    Une courbe fermée $\Gamma$ sépare le plan en deux ensembles: un intérieur borné et un extérieur non-borné \textit{(fun fact, cette proposition, appelée le théorème de Jordan, est connue pour la difficulté de sa démonstration)}.

    Ainsi, on note $\text{int}\left(\Gamma\right)$ \important{l'intérieur} de $\Gamma$ qui est l'ouvert borné dont le bord est confondu avec $\Gamma$. 
}

\parag{Définition: Courbe régulière par morceaux}{
    Une courbe $\Gamma$ est \important{régulière par morceaux} s'il existe des courbes régulières $\Gamma_1, \ldots, \Gamma_n$ de paramétrisations $\gamma_i : \left[a_i, b_i\right] \mapsto \Gamma_i$ telles que: 
    \[\Gamma = \bigcup_{i=1}^{n} \Gamma_i\]
    
    De plus, les $\Gamma_i$ doivent être ``connectées'': 
    \[\gamma_i\left(b_i\right) = \gamma_{i+1}\left(a_{i+1}\right), \mathspace \forall i = 1, \ldots, n-1\]

    \subparag{Intuition}{
        Une courbe régulière par morceaux a une paramétrisation dérivable partout, sauf aux points de jonctions.
    }
}

\parag{Exemple 1}{
    Considérons le segment entre $1$ et $1+i$ qu'on note: 
    \[\Gamma = \left[1; 1 + i\right] = \left[1+i; 1\right]\]
    
    Nous pouvons considérer deux paramétrisations, pour parcourir la courbe dans un sens ou dans l'autre:
    \[\gamma_1\left(t\right) = 1 + it, \mathspace t \in \left[0, 1\right]\] 
    \[\gamma_2\left(t\right) = 1 + i\left(1 - t\right), \mathspace t \in \left[0, 1\right]\]
    
    Cela nous montre que, dans le plan complexe comme dans $\mathbb{R}^n$, les courbes peuvent avoir un sens de parcours différent.
}

\parag{Example 2}{
    Soient $z_1, z_2 \in \mathbb{C}$ quelconques. Nous considérons le segment entre $z_1$ et $z_2$, noté: 
    \[\Gamma = \left[z_1, z_2\right] = \left[z_2, z_1\right]\]
    
    Alors, nous avons les formules générales suivantes pour les paramétrisations de $\Gamma$: 
    \[\gamma_1\left(t\right) = z_1 \left(1 -t\right) + z_2 t, \mathspace t \in \left[0, 1\right]\]
    \[\gamma_2\left(t\right) = z_2\left(1- t\right) + z_1 t, \mathspace t \in \left[0, 1\right]\]

    \subparag{Remarque}{
        Lorsqu'on a des cas plus simple, comme dans le cas de l'exemple précédent, il est mieux de ne pas appliquer bêtement cette formule et d'y aller plus intuitivement.
    }
}

\parag{Example 3}{
    Soit $\Gamma$ le cercle de centre $z_0 \in \mathbb{C}$ et de rayon $R \in \mathbb{R}_{> 0}$. 

    Nous pouvons simplement considérer la paramétrisation suivante:
    \[\gamma\left(t\right) = z_0 + R\cos\left(t\right) + Ri\sin\left(t\right) = z_0 + Re^{it}, \mathspace t \in \left[0, 2\pi\right]\]

    En particulier, si $z_0 = 0$ et $R = 1$, alors $\Gamma$ est le cercle unité.

    \subparag{Remarque}{
        Cette formule est très importante.
    }
}

\parag{Définition: Intégrale complexe}{
    Soit $\Omega \subset \mathbb{C}$ un ensemble ouvert, $f: \Omega \mapsto \mathbb{C}$ une fonction continue, et $\Gamma \subset \Omega$ une courbe.

    Si $\Gamma \subset \Omega$ est une courbe régulière de paramétrisation $\gamma : \left[a, b\right] \mapsto \Gamma$, alors, \important{l'intégrale} de $f$ le long de $\Gamma$ est le nombre: 
    \[\int_{\Gamma} f dz \over{=}{déf} \int_{a}^{b} f\left(\gamma\left(t\right)\right) \cdot \gamma'\left(t\right)dt\]
    
    Si $\Gamma = \Gamma_1 \cup \ldots \Gamma_n$ est une courbe régulière par morceaux où $\Gamma_i \subset \Omega$ pour tout $i$, alors: 
    \[\int_{\Gamma} f dz = \sum_{i=1}^{n} \int_{\Gamma_i} f dz\]

    \subparag{Remarque}{
        Comme nous le verrons, le signe de ces intégrales dépend du sens de parcours.
    }
}

\parag{Exemple 1}{
    Soient $f\left(z\right) = z$ et la courbe $\Gamma = \left[1; 1 + i\right]$. 

    Nous choisissons la paramétrisation suivante: 
    \[\gamma_1\left(t\right) = 1 + it, \mathspace t \in \left[0, 1\right]\]

    Cela nous donne ainsi: 
    \[\int_{\Gamma} f \cdot dz = \int_{0}^{1} \left(1 + it\right)\left(1 + it\right)' dt = \int_{0}^{1} \left(1 + it\right)i dt = \int_{0}^{1} \left(i - t\right)dt = \left[it - \frac{t^2}{2}\right]_{0}^{1} = -\frac{1}{2} + i\]
    

    Cependant, choisissons maintenant la paramétrisation qui va dans l'autre sens: 
    \[\gamma_2\left(t\right) = 1 + i\left(1 -t\right) \implies \gamma_2'\left(t\right) = -i\]
    
    Alors, on trouve: 
    \[\int_{\Gamma} f \cdot dz = \int_{0}^{1} \left(1 + i - it\right) \left(-i\right)dt = \int_{0}^{1} \left(-i + 1 -t\right)dt = \left[-it + t - \frac{t^2}{2}\right]_0^1 = \frac{1}{2} - i\]

    Nous voyons donc que le signe du résultat dépend du sens de parcours, comme attendu.
}

\parag{Exemple 2}{
    Considérons la fonction $f\left(z\right) = \frac{1}{z}$ et le cercle unité $\Gamma$.

    Nous prenons la paramétrisation suivante, qui parcours la courbe dans le sens trigonométrique: 
    \[\gamma\left(t\right) = e^{it}, \mathspace t \in \left[0, 2\pi\right]\]
    
    Nous obtenons que $\gamma'\left(t\right) = ie^{it}$, donc: 
    \[\int_{\Gamma} f dz = \int_{0}^{2\pi} \frac{1}{e^{it}} ie^{it} dt = i \int_{0}^{2\pi} dt = 2\pi i\]

    Cette valeur est très importante, et elle va revenir dans un théorème très bientôt.

    \subparag{Remarque}{
        Comme nous l'avons vu en Analyse 3, une intégrale sur une courbe fermée ne donne pas toujours 0.
    }
}

\parag{Définition: Orientation}{
    Une courbe régulière fermée $\Gamma$ de paramétrisation $\gamma : \left[a, b\right] \mapsto \mathbb{C}$ est \important{orientée positivement} si elle ``laisse l'intérieur à gauche''.


    \subparag{Example}{
        Par exemple, nous pouvons considérer la courbe suivante, dont la flèche montre l'orientation positive:
        \svghere[0.6]{CourbeOrienteePositivement.svg}
    }
}

\parag{Théorème de Cauchy}{
    Soit $\Omega \subset \mathbb{C}$ ouvert, $f: \Omega \mapsto \mathbb{C}$ holomorphe, et $\Gamma$ une courbe fermée et régulière par morceaux telle que $\text{int}\left(\Gamma\right) \cup \Gamma \subset \Omega$.

    Alors, le théorème de Cauchy dit que: 
    \[\int_{\Gamma} fdz = 0\]

    \subparag{Remarque}{
        L'hypothèse que $\text{int}\left(\Gamma\right) \cup \Gamma \subset \Omega$ est parfois décrite en demandant que $\Omega$ est simplement connexe; que ce domaine n'a ``pas de trou''.
    }
    
    \subparag{Réciproque}{
        La réciproque de ce théorème est vraie, et est appelée le théorème de Morera. 

        En d'autres mots, si $\int_{\Gamma} f\left(z\right) = 0$ pour toute courbe $\Gamma \subset \Omega$ fermée, alors $f$ est holomorphe.
    }

    \subparag{Preuve}{
        Soit $\Gamma$ une courbe quelconque, que nous pouvons écrire: 
        \[\Gamma = \left\{\gamma\left(t\right) = \alpha\left(t\right) + i\beta\left(t\right), t \in \left[a, b\right]\right\} \subset \mathbb{C}\]

        Nous considérons son équivalent dans $\mathbb{R}^2$: 
        \[\widetilde{\Gamma} = \left\{\widetilde{\gamma}\left(t\right) = \left(\alpha\left(t\right), \beta\left(t\right)\right), t \in \left[a, b\right]\right\} \subset \mathbb{R}^2\]
        
        Nous notons de plus: 
        \[f\left(z\right) = f\left(x + iy\right) = u\left(x, y\right) + iv\left(x, y\right)\]
        
        Nous utilisons ceci pour calculer notre intégrale: 
        \autoeq[s]{\int_{\Gamma} fdz = \int_{a}^{b} f\left(\gamma\left(t\right)\right)\gamma'\left(t\right) dt = \int_{a}^{b} \left(u\left(\alpha\left(t\right), \beta\left(t\right)\right) + iv\left(\alpha\left(t\right), \beta\left(t\right)\right)\right)\left(\alpha'\left(t\right) + i\beta'\left(t\right)\right) =
        \int_{a}^{b} \left(u\left(\alpha, \beta\right)\alpha' - v\left(\alpha, \beta\right)\beta'\right)dt + i\int_{a}^{b} \left(u\left(\alpha, \beta\right)\beta' + v\left(\alpha, \beta\right)\alpha'\right)dt = \int_{a}^{b} \begin{pmatrix} u\left(\alpha, \beta\right) \\ -v\left(\alpha, \beta\right) \end{pmatrix} \dotprod \begin{pmatrix} \alpha' \\ \beta' \end{pmatrix} dt + i\int_{a}^{b} \begin{pmatrix} v\left(\alpha, \beta\right) \\ u\left(\alpha, \beta\right)  \end{pmatrix} \dotprod \begin{pmatrix} \alpha' \\ \beta' \end{pmatrix} dt
    = \int_{a}^{b} \begin{pmatrix} u\left(\widetilde{\gamma}\right) \\ -v\left(\widetilde{\gamma}\right) \end{pmatrix} \dotprod \gamma' dt + i \int_{a}^{b} \begin{pmatrix} v\left(\widetilde{\gamma}\right) \\ u\left(\widetilde{\gamma}\right) \end{pmatrix} \dotprod \gamma' dt
    = \int_{\widetilde{\Gamma}} \left(u, -v\right) \dotprod d \ell + i \int_{\widetilde{\Gamma}} \left(v, u\right) \dotprod d \ell }

    Puisque $f$ est holomorphe, $u$ et $v$ sont de classe $C_1$. De plus, puisque $\text{int}\left(\Gamma\right) \subset \Omega$, nous savons que $\text{int}\left(\widetilde{\Gamma}\right)$ est simplement connexe. Nous avons donc les hypothèses pour appliquer le théorème de Green: 
    \autoeq{\int_{\Gamma} fdz = \int_{\text{int}\left(\widetilde{\Gamma}\right)} \rot\left(u, -v\right)dxdy + i \int_{\text{int}\left(\widetilde{\Gamma}\right)} \rot\left(v, u\right)dxdy = \int_{\text{int}\left(\widetilde{\Gamma}\right)} \underbrace{\left(-v_x - u_y\right)}_{= 0}dxdy + i \int_{\text{int}\left(\widetilde{\Gamma}\right)} \underbrace{\left(u_x - v_y\right)}_{= 0} dxdy = 0}
    par les équations de Cauchy Riemann.

    \qed
    }
}

\parag{Théorème: Formule intégrale de Cauchy (FIC)}{
    Soit $\Omega \subset \mathbb{C}$ ouvert, $f: \Omega \mapsto \mathbb{C}$ holomorphe, et $\Gamma$ une courbe fermée et régulière par morceaux telle que $\text{int}\left(\Gamma\right) \cup \Gamma = \bar{\text{int}\left(\Gamma\right)} \subset \Omega$.

    Si de plus $\gamma$ est orienté positivement, et $z_0 \in \text{int}\left(\Gamma\right)$, alors la formule intégrale de Cauchy (FIC) dit que, pour tout $n \in \mathbb{N}$:
    \[\int_{\Gamma} \frac{f\left(z\right)}{\left(z - z_0\right)^{n+1}} dz = 2\pi i \frac{f^{\left(n\right)}\left(z_0\right)}{n!}\]

    Le $z_0$ est appelé une \important{singularité}. L'intégrale ne dépend que de la présence de cette singularité dans l'intérieur de la courbe.

    Il est important de remarquer que, même si $f\left(z\right)$ est holomorphe, alors $\frac{f\left(z\right)}{\left(z - z_0\right)^{n+1}}$ ne l'est pas à cause de sa singularité.

    \subparag{Cas $n = 0$}{
        Dans le cas où $n = 0$, nous avons:
        \[\int_{\Gamma} \frac{f\left(z\right)}{z-z_0}dz = 2\pi i f\left(z_0\right)\]
    }

    \subparag{Utilisation}{
        Dès qu'on voit une singularité à l'intérieur de la courbe d'intégration, nous pouvons directement essayer d'utiliser ce théorème.
    }

    \subparag{Remarque}{
        Ce théorème et le précédent montrent un des supers-pouvoirs \textit{[sic]} des fonctions holomorphes.
    }

    \subparag{Esquisse de preuve}{
        Pour faire cette preuve, il faut commencer par démontrer le cas $n = 0$, puis l'appliquer pour démontrer le cas général. Nous n'allons pas faire ce premier point, mais uniquement le deuxième. \frownie

        Nous supposons donc que l'égalité suivante est démontrée: 
        \[f\left(z_0\right) = \frac{1}{2\pi i} \int_{\Gamma} \frac{f\left(z\right)}{z - z_0} dz\]

        Nous allons supposer que l'opérateur dérivée et l'opérateur intégrale commutent (ce qu'on peut démontrer, dans ce cas, à l'aide de la définition de la dérivée; et ce qu'on devrait faire pour que la preuve soit formelle): 
        \autoeq{f'\left(z_0\right) = \left(\frac{1}{2\pi i} \int_{\Gamma} \frac{f\left(z\right)}{z - z_0} dz\right)' = \frac{1}{2\pi i} \int_{\Gamma} f\left(z_0\right) \left(\frac{1}{z - z_0}\right)' = \frac{1}{2\pi i} \int_{\Gamma} f\left(z\right) \frac{-1}{\left(z- z_0\right)^2}\left(-1\right) dz = \frac{1}{2\pi i} \int_{\Gamma} \frac{f\left(z\right)}{\left(z - z_0\right)^2}dz}
        
        Nous avons démontré que le cas $n = 0$ implique le cas $n = 1$. Les autres se démontrent par dérivées successives.
    }
}

\parag{Corollaire}{
    Si $f$ est holomorphe, alors $f$ est infiniment dérivable.

    \subparag{Intuition}{
        Ce corollaire découle directement de la FIC. En effet, nous avons vu que nous pouvions écrire: 
        \[f^{\left(n\right)}\left(z_0\right) = \frac{n!}{2\pi i} \int_{\Gamma} \frac{f\left(z\right)}{\left(z - z_0\right)^{n+1}} dz\]
        
        Ainsi, n'importe quelle dérivée peut être calculée à partir de la fonction de départ; toutes ces fonctions existent.
    }
    
}


\parag{Exemple 1}{
    Nous voulons vérifier le théorème de Cauchy pour $f\left(z\right) = z^2$ sur le cercle unité $\Gamma$.

    Nous avons simplement: 
    \[\int_{\Gamma} z^2 dz = \int_{0}^{2\pi} \left(e^{it}\right)^2 e^{it} i dt = i \int_{0}^{2\pi} e^{3it}dt = \left[\frac{e^{3it}}{3}\right]_{0}^{2\pi} = \frac{e^{i6\pi} - e^{0}}{3} = \frac{1 - 1}{3} = 0\]

    Le théorème est donc vérifié dans ce cas.
}

\parag{Exemple 2}{
    Soient $f\left(z\right) = \frac{1}{z}$ et le cercle unitaire $\Gamma$. Nous voulons calculer l'intégrale de $f$ sur cette courbe en utilisant la FIC.

    Nous pouvons choisir $g\left(z\right) = 1$ (qui est bien holomorphe sur un ouvert contenant $\bar{\text{int}\left(\Gamma\right)}$) et $z_0 = 0 \in \text{int}\left(\Gamma\right)$, ce qui nous donne:
    \[\int_{\Gamma} \frac{1}{z} dz = \int_{\Gamma} \frac{g\left(z\right)}{z - z_0} dz \over{=}{FIC} 2\pi i g\left(z_0\right) = 2\pi i\]

    Nous voyons bien que la FIC peut nous sauver beaucoup de temps.
}

\end{document}
