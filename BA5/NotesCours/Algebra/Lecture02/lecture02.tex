% !TeX program = lualatex
% Using VimTeX, you need to reload the plugin (\lx) after having saved the document in order to use LuaLaTeX (thanks to the line above)

\documentclass[a4paper]{article}

% Expanded on 2023-10-02 at 15:19:49.

\usepackage{../../style}

\title{Algebra}
\author{Joachim Favre}
\date{Lundi 02 octobre 2023}

\begin{document}
\maketitle

\lecture{2}{2023-10-02}{AICC 2}{
    \begin{itemize}[left=0pt]
    \item Explanation of the RSA cryptosystem.
    \item Definition of groups and order.
    \item Definition of subgroups.
    \item Definition of left cosets, and of index.
    \item Proof of Lagrange's theorem, and some corollaries.
\end{itemize}
    
}

\section{Groups}

\begin{parag}{RSA}
    Groups are very useful for the RSA cryptosystem. The algorithm goes as follows:
    \begin{enumerate}
        \item We choose two large distinct primes $p, q$.
        \item Let $m = pq$. We notice that we can easily compute $\phi\left(m\right) = \left(p-1\right)\left(q-1\right)$.
        \item We choose some $e \leq m-1$ such that $\gcd\left(e, \phi\left(m\right)\right) = 1$.
        \item We use the Euclidean algorithm to find a $d \leq m-1$ such that $ed - k \phi\left(m\right) = 1$ for some $k \in \mathbb{Z}$.
        \item We finally publish the encoding key $\left(m, e\right)$ and keep secret the decoding key $\left(m, d\right)$.
    \end{enumerate}
    
    Let's say that Bob wants to send a secret message to Alice. The message is a number $x$ such that $0 \leq x \leq m - 1$.

    Alice previously published the encoding key $\left(m, e\right)$ and kept the decoding key $\left(m, d\right)$. Bob uses the public key to compute $y = x^e \Mod m$, and sends $y$ publicly to Alice. It is very hard to inverse this function (this problem is called the Discrete logarithm problem), so this is secure. Alice can finally recover $x = y^d \Mod m$. Other people cannot do that since it is very hard to compute $d$ from $m$, because it is hard to compute $\phi\left(m\right)$ (this comes from the difficulty to compute the prime factorisation of a number).
    
    To show that this algorithm works, we need to show that: 
    \[x^{ed} \Mod m = x, \mathspace \forall x \in \left\{0, 1, \ldots, m-1\right\}\]
    
    This is done by group theory.
\end{parag}

\subsection{Definitions}

\begin{parag}{Definition: Group}
    A \important{group} is a non-empty set $G$ together with a binary operation $\cdot: G \times G \mapsto G$, satisfying the following axioms:
    \begin{itemize}
        \item (Associativity) For any $a, b, c \in G$:
        \[\left(a\cdot b\right)\cdot c = a\cdot \left(b\cdot c\right)\]
        \item (Existence of a neutral element) There exists some $1 \in G$, such that, for any $a \in G$:
        \[1\cdot a = a\cdot 1 = a\]
        \item (Existence of the inverse) For any $a \in G$, there exists some $a^{-1} \in G$ such that:
        \[a^{-1} \cdot a = a\cdot a^{-1} = 1\]
    \end{itemize}

    We write groups $G$, $\left(G, \cdot \right)$ or $\left(G, \cdot , 1\right)$.
\end{parag}

\begin{parag}{Definition: Order}
    Let $\left(G, \cdot \right)$ be a group. 

    It is said to be \important{finite} if $\left|G\right| < \infty$. In this case, we name $\left|G\right|$ to be the order of the group.
\end{parag}

\begin{parag}{Definition: Abelian group}
    Let $\left(G, \cdot \right)$ be a group.

    It is said to be \important{Abelian} (or commutative) if it also follows the following property, for any $a, b \in G$: 
    \[a\cdot b = b\cdot a\]
\end{parag}

\begin{parag}{Examples}
    We have the following examples:
    \begin{enumerate}
        \item $\left(\mathbb{R}, +, 0\right)$ is an infinite Abelian group. For any $x \in \mathbb{R}$, the inverse in this group is $-x$.
        \item $\left(\mathbb{Z}, +, 0\right)$ is another infinite Abelian group.
    \end{enumerate}
\end{parag}

\begin{parag}{Definition: Integers modulo $n$}
    The set of integers modulo $n$ is written $\mathbb{Z} / n\mathbb{Z}$. Its elements are equivalence classes of the equivalence relation of numbers modulo $n$: 
    \[\mathbb{Z} / n \mathbb{Z} = \left\{\left[0\right], \ldots, \left[n-1\right]\right\}\]
\end{parag}

\begin{parag}{Definition: Cyclic group of order $n$}
    We notice that $\left(\mathbb{Z} / n\mathbb{Z}, +_n\right)$ where $+_n$ is the addition modulo $m$, is an Abelian group of order $n$. The inverse of an element $\left[k\right]$ is $\left[n-k\right]$.

    This is named the \important{cyclic group of order $n$}, written $C_n$

    \begin{subparag}{Example}
        For instance, in $\mathbb{Z} / 12 \mathbb{Z} = \left\{\left[0\right], \ldots, \left[11\right]\right\}$, we have: 
        \[\left[7\right] + \left[5\right] = \left[12\right], \mathspace \left[10\right] + \left[3\right] = \left[1\right]\]
    \end{subparag}
\end{parag}

\begin{parag}{Remark}
    All sets are not Abelian. Indeed, let us consider the group $GL_{n}\left(\mathbb{R}\right)$, which is the set of real $n \times n$ matrices with non-zero determinant with respect to matrix multiplication.

    There is indeed a neutral element---the identity matrix---the operation is indeed associative, and they all have inverses. However, this group is non-abelian since matrix multiplication is not commutative in general. 
\end{parag}

\begin{parag}{Proposition}
    Let $n \in \mathbb{Z}^*$ and $a \in \mathbb{Z}$. 

    $a$ has an inverse with respect to the multiplication modulo $n$ if and only if: 
    \[\gcd\left(a, n\right) = 1\]
    
    \begin{subparag}{Proof}
        The existence of some $x \in \mathbb{Z}$ such that $ax \Mod n = 1$ is equivalent to the existence of $x, y \in \mathbb{Z}$ such that $ax + ny = 1$. However, by Bézout's theorem, this is equivalent to: 
        \[\gcd\left(a, n\right) = 1\]

        \qed
    \end{subparag}
\end{parag}


\begin{parag}{Observation}
    Let $G = \left(\mathbb{Z}/{n}\mathbb{Z}, \cdot, 1\right)^*$ be the group of invertible numbers with respect to the multiplication mod $n$. By our proposition, this is of the form:
    \[G = \left\{1 \leq x \leq n | \gcd\left(x, n\right) = 1\right\}\]

    Thus, by definition of $\phi\left(n\right)$, this yields that the order of $G$ is: 
    \[\left|G\right| = \phi\left(n\right)\]
    
    We thus see that this is a finite Abelian group of order $\phi\left(n\right)$, and with identity $\left[1\right]$.
\end{parag}

\subsection{Subgroups and cosets}

\begin{parag}{Definition: Subgroup}
    Let $\left(G, \cdot, 1 \right)$ be a group, and $H \subset G$ be a subset of $G$.

    $H$ is a \important{subgroup} of $G$ if $H$ is such that:
    \begin{enumerate}
        \item $1 \in H$
        \item For any $a, b \in H$, then $a\cdot b \in H$.
        \item For any $a \in H$, then $a^{-1} \in H$.
    \end{enumerate}
\end{parag}

\begin{parag}{Proposition}
    Let $G$ be a group and $H \subset G$ be a subgroup. 

    Then, $H$ is a group.
\end{parag}

\begin{parag}{Example 1}
    We noticed that $\left(GL_{n}\left(\mathbb{R}\right), \cdot \right)$ is a group. 

    Now, we can take $H \subset GL_{n}\left(\mathbb{R}\right)$ to be the set of invertible diagonal matrices: 
    \[D = \left\{\begin{pmatrix} a & 0 \\ 0 & b \end{pmatrix} | a, b \in \mathbb{R}^*\right\}\]
    
    The identity matrix is indeed in $D$. Moreover, multiplications and inverses are closed under $D$: 
    \[\begin{pmatrix} a_1 & 0 \\ 0 & b_1 \end{pmatrix} \begin{pmatrix} a_2 & 0 \\ 0 & b_2 \end{pmatrix} = \begin{pmatrix} a_1 a_2 & 0 \\ 0 & b_1 b_2 \end{pmatrix} \in D\] 
    \[\begin{pmatrix} a & 0 \\ 0 & b \end{pmatrix}^{-1} = \begin{pmatrix} \frac{1}{a} & 0 \\ 0 & \frac{1}{a} \end{pmatrix} \in D\]

    Thus, $H$ is a subgroup of $GL_n\left(\mathbb{R}\right)$.
\end{parag}

\begin{parag}{Example 2}
    We notice that $\left(3 \mathbb{Z}, +, 0\right) \subset \left(\mathbb{Z}, +, 0\right)$ is a subgroup.
\end{parag}

\begin{parag}{Definition: Subgroup generated by one element}
    Let $G$ be a group, and $g \in G$.

    The \important{subgroup generated by $g$} is defined as: 
    \[\left\langle g \right\rangle = \left\{1, g, g^2, \ldots\right\} \cup \left\{g^{-1}, g^{-2}, \ldots\right\}\]
    where $g^{i+k} = g^i \cdot g^k$ is the operation $\cdot $ repeated on $i+k$ $g$'s.
    
    \begin{subparag}{Property}
        $\left\langle g \right\rangle \subset G$ is a subgroup by construction.
    \end{subparag}

    \begin{subparag}{Remark}
        If $g$ has a finite order $k$, then we just have: 
        \[\left\langle g \right\rangle = \left\{1, g, g^2, \ldots, g^{k-1}\right\}\]
        
        Indeed, we for instance have that $g^{k-1} = g^{-1}$ since: 
        \[g^{k-1} g = g^{k} = 1\]
    \end{subparag}
\end{parag}

\begin{parag}{Example}
    Let us consider the group $G = \left(\mathbb{Z}, +, 0\right)$, and $g = 3 \in \mathbb{Z}$. Then, this generates: 
    \[\left\langle g \right\rangle = \left\{0, \pm 3, \pm 6, \ldots\right\} = 3 \mathbb{Z}\]
\end{parag}

\begin{parag}{Definition: Order of an element}
    Let $\left(G, \cdot \right)$ be a group. If there exists some minimal $n \in \mathbb{N}^*$ such that $g^n = 1$, then $n$ is called the \important{order} of $g$.

    \begin{subparag}{Remark}
        This must not be mistaken with the order of the group.
    \end{subparag}

    \begin{subparag}{Observation}
        In a finite group, each element has a finite order by the pigeon-hole principle. If however the group is infinite, some elements might be such that $g^n \neq 1$ for all $n$.
    \end{subparag}
\end{parag}

\begin{parag}{Definition: Left coset}
    Let $G$ be a group, $H \subset G$ be a subgroup and $g \in G$. 

    The \important{left coset $gH$} (the left coset of $g$ with respect to $H$) is the set of group elements of the form: 
    \[gH = \left\{gh | h \in H\right\}\]
    
    \begin{subparag}{Remark}
        This is not necessarily a subgroup, since there might not be the identity element for instance.
    \end{subparag}
\end{parag}

\begin{parag}{Properties}
    Let $G$ be a finite group, and $H \subset G$ be a subgroup of $G$.

    We have the following properties:
    \begin{enumerate}
        \item Two left cosets $xH$ and $yH$ are either equal or disjoint: 
        \[x H = y H \mathspace \text{ or } \mathspace xH \cap y H = \o\]
        \item Any $g \in G$ belongs to a left $H$-coset.
        \item For any $x \in G$, then $\left|x H\right| = \left|H\right|$.
    \end{enumerate}
    
    \begin{subparag}{Proof 1}
        Let's suppose that we have $x H \cap y H \neq \o$.

        Then, there exist $h_1, h_2 \in H$ such that $x h_1 = y h_2$. Since elements of a group are invertible, we get: 
        \[x = y \underbrace{h_2 h_1^{-1}}_{\in H} = y h_3\]
        
        This means that any element of $xH$ can be written as: 
        \[xh = y \underbrace{h_3 h}_{\in H} = y h_4 \in y H\]

        Since any element $xh \in xH$ is such that $xh \in yH$, this yields that $x H \subset y H$. By starting the proof again but using a symmetrical argument, we can get that $yH \subset x H$. Combining both together, we indeed find $x H = yH$.
    \end{subparag}
    
    \begin{subparag}{Proof 2}
        Let $g \in G$. We consider the following coset: 
        \[gH = \left\{g\cdot 1, g\cdot h_1, g\cdot h_2, \ldots\right\}\]
        
        However, $g = g\cdot 1$; so $g \in gH$.
    \end{subparag}
    
    \begin{subparag}{Proof 3}
        To show that two sets have the same cardinality, a good way is to find a bijection $f: H \mapsto xH$. Thus, let: 
        \[f\left(h\right) = xh\]
        
        We first notice that $f$ is surjective since we reach any element of $xH$ by definition of cosets. It is moreover injective since: 
        \[x h_1 = x h_2 \iff x^{-1} x h_1 = x^{-1} x h_2 \iff h_1 = h_2\]
        
        This yields that $f$ is bijective, and thus that $\left|H\right| = \left|xH\right|$ for any $x \in G$.

        \qed
    \end{subparag}
\end{parag}

\begin{parag}{Example}
    Let us consider the group $\left(\mathbb{Z}, +, 0\right)$ and the subgroup $3 \mathbb{Z} \subset \mathbb{Z}$.

    The coset of 0 with respect to $3\mathbb{Z}$ in $\mathbb{Z}$ is: 
    \[\left\{0 + 3k\right\}_{k \in \mathbb{Z}} = 3 \mathbb{Z}\]
    
    The coset of 1 with respect to $3\mathbb{Z}$ in $\mathbb{Z}$ is: 
    \[\left\{1 + 3k\right\}_{k \in \mathbb{Z}} = \left\{1, -2, 4, -5, 7, \ldots\right\}\]
    which the set of all numbers which rest after a division by 3 is 1.
    
    We notice that the coset of 10 with respect to $3\mathbb{Z}$ in $\mathbb{Z}$ is exactly the same coset:
    \[\left\{10 + 3k\right\}_{k \in \mathbb{Z}} = \left\{1, -2, 4, -5, 7, \ldots\right\}\]
\end{parag}

\begin{parag}{Definition: Index}
    The number of left $H$-cosets in $G$ is called the \important{index} of $H$ in $G$, written $\left[G:H\right]$. 
\end{parag}


\begin{parag}{Lagrange's theorem}
    Let $G$ be a finite group, and $H \subset G$ be a subgroup.

    Then, $\left|H\right|$ divides $\left|G\right|$; and:
    \[\frac{\left|G\right|}{\left|H\right|} = \left[G:H\right]\]

    \begin{subparag}{Proof}
        We know that each $g \in G$ belongs to a left $H$-coset and that either $x H = yH$ or $xH \cap yH = \o$. We can thus write $G = \bigcup_{i=1}^{r} x_i H$ as a disjoint union of finitely many left $H$-cosets.

        This allows us to compute its cardinality: 
        \[\left|G\right| = \sum_{i=1}^{r} \left|x_i H\right|\]
        
        Now, we now that $\left|x_i H\right| = \left|H\right|$, so: 
        \[\left|G\right| = \sum_{i=1}^{r} \left|H\right| = r\left|H\right|\]

        We indeed get that $\left|H\right|$ divides $\left|G\right|$. However, $r$ is the number of cosets, so this is the index of $H$ in $G$, yielding: 
        \[\left[G:H\right] = r = \frac{\left|G\right|}{\left|H\right|}\]
        
        \qed
    \end{subparag}
\end{parag}

\begin{parag}{Corollary 1}
    Let $G$ be a finite group and $g \in G$.

    Then, the order of $g$ divides $\left|G\right|$.

    \begin{subparag}{Proof}
        Let us consider the subgroup generated by $g$: 
        \[H = \left\langle g \right\rangle = \left\{1, g, \ldots, g^{k-1}\right\}\]
        where $k$ is the order of $g$.

        We know that $H \subset G$ is a subgroup. Thus, by Lagrange's theorem, $\left|H\right|$ divides $\left|G\right|$. However, $\left|\left\langle g \right\rangle\right| = k$, so we indeed got that the order of $g$ divides $\left|G\right|$.

        \qed
    \end{subparag}
    
\end{parag}

\begin{parag}{Corollary 2}
    Let $G$ be a finite group, and $g \in G$.

    Then, $g^{\left|G\right|} = 1$.

    \begin{subparag}{Proof}
        Let $k$ be the order of $g$. We know that $k$ divides $\left|G\right|$, so there exists some integer $t$ such that: 
        \[\left|G\right| = kt\]

        This yields that: 
        \[g^{\left|G\right|} = \left(g^k\right)^t = 1^t = 1\]

        \qed
    \end{subparag}
\end{parag}

\begin{parag}{Euler's theorem}
    Let $a, n \in \mathbb{Z}_+$ such that $\gcd\left(a, n\right) = 1$.

    Then: 
    \[\congruent{a^{\phi\left(n\right)}}{1}{n}\]
    
    \begin{subparag}{Proof}
        Let $G = \left(\mathbb{Z}/n\mathbb{Z}, \cdot, 1\right)^*$. We saw that $\left|G\right| = \phi\left(n\right)$.

        We know that if $\gcd\left(a, n\right) = 1$, then $\left[a\right] \in G$. This tells us by our second corollary that: 
        \[\left[a\right]^{\phi\left(n\right)} = \left[1\right]\]
        
        In other words:
        \[\congruent{a^{\phi\left(n\right)}}{1}{n}\]

        \qed
    \end{subparag}
\end{parag}

\begin{parag}{Fermat's little theorem}
    Let $a \in \mathbb{Z}_+$ and $p$ be a prime such that $p$ does not divide $a$.

    Then: 
    \[\congruent{a^{p-1}}{1}{p}\]
    
    \begin{subparag}{Proof}
        We notice that, by hypothesis, $\gcd\left(a, p\right) = 1$. So, by Euler's theorem: 
        \[\congruent{a^{\phi\left(p\right)}}{1}{p} \iff \congruent{a^{p-1}}{1}{p}\]
        
        \qed
    \end{subparag}
\end{parag}


\end{document}
