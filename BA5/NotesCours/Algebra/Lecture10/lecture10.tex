% !TeX program = lualatex
% Using VimTeX, you need to reload the plugin (\lx) after having saved the document in order to use LuaLaTeX (thanks to the line above)

\documentclass[a4paper]{article}

% Expanded on 2023-12-05 at 23:27:40.

\usepackage{../../style}

\title{Algebra}
\author{Joachim Favre}
\date{Mardi 05 décembre 2023}

\begin{document}
\maketitle

\lecture{10}{2023-11-27}{Generalising Euclidean division}{
\begin{itemize}[left=0pt]
    \item Proof of the Chinese remainder theorem.
    \item Proof of the CRT for integers, and examples on how to use it.
    \item Definition of polynomial rings, and proof of some properties.
    \item Definition of Euclidean domain, and proof of some properties.
\end{itemize}

}

\subsection{Chinese remainder theorem}

\begin{parag}{Chinese remainder theorem (CRT)}
    Let $A$ be a commutative ring, and $I, J \subset A$ be ideals such that $I + J = A$.

    Then, there exists a ring isomorphism defined by:
    \[\begin{split}
    f:  A / \left(I \cap J\right) &\longmapsto A / I \times A / J \\
    f\left(\left[x\right]_{I \cap J}\right) &\longmapsto \left(\left[x\right]_{I}, \left[x\right]_{J}\right)
    \end{split}\]
    
    This proposition is known as the \important{Chinese remainder theorem} (CRT).

    \begin{subparag}{Proof}
        First, we notice that the map $g: \left[x\right]_{I \cap J} \mapsto \left[x\right]_I$ is a ring homomorphism: it preserves all ring operations. This therefore implies that $f: x \mapsto \left(\left[x\right]_I, \left[x\right]_J\right)$ is also a ring homomorphism.

        We now want to show that it is bijective. We begin by showing that $f$ is surjective. Let $a_1, a_2 \in A$. We want to show that there exists some $a \in A$ such that $\congruent{a}{a_1}{I}$ and $\congruent{a}{a_2}{J}$. Since $I + J = A$, we have that $a_1 - a_2 \in A$ can be written as: 
        \[a_1 - a_2 = -i + j \iff a_1 + i = a_2 + j \over{=}{def} a\]
        for some $i \in I$ and $j \in J$. We have indeed got that $\congruent{a}{a_1}{I}$ and $\congruent{a}{a_2}{J}$, meaning that $f$ is surjective.

        We now want to show that $f$ is injective. We keep our $a = a_1 + i = a_2 + j \in A$, and we let $b \in A$ such that $\congruent{b}{a_1}{I}$ and $\congruent{b}{a_2}{J}$. Then, by definition of quotient rings, there exists $i' \in I$ and $j' \in J$ such that: 
        \[b = a_1 + i' = a_2 + j' \implies a - b = \underbrace{i - i'}_{\in I} = \underbrace{j - j'}_{\in J} \in I \cap J\]
        
        Therefore, $\left[a\right]_{I \cap J} = \left[b\right]_{I \cap J}$, meaning that $f$ is indeed injective.

        Since $f$ is both surjective and injective, it is bijective. Since it is also a ring homomorphism, it is a ring isomorphism.

        \qed
    \end{subparag}
\end{parag}

\begin{parag}{Corollary: CRT for integers}
    Let $n, m \in \mathbb{Z}$ be coprime, i.e $\gcd\left(n, m\right) = 1$.

    Then, for any $a_1, a_2 \in \mathbb{Z}$, there exists some $a \in A$ such that: 
    \[\begin{systemofequations} \congruent{a}{a_1}{n}\\ \congruent{a}{a_2}{m} \end{systemofequations}\]
    
    The full set of solutions of this pair of congruences is $\left\{a + nm \mathbb{Z}\right\}$.

    \begin{subparag}{Proof}
        We know by hypothesis that $\gcd\left(n, m\right) = 1$. This implies by Bézout's theorem that there exists $x, y \in \mathbb{Z}$ such that: 
        \[xn + ym = 1 \implies n\mathbb{Z} + m\mathbb{Z} = \mathbb{Z}\]
        since an ideal which contains $1$ is non-proper.

        By the CRT, we know that $\mathbb{Z} / \left(\left(n\right) \cap \left(m\right)\right) \simeq \mathbb{Z}/n\mathbb{Z} \times \mathbb{Z}/m\mathbb{Z}$. Thus, for any pair $\left(\left(a_1 \Mod n\right), \left(a_2 \Mod m\right)\right) \in \mathbb{Z}/n\mathbb{Z} \times \mathbb{Z}/m\mathbb{Z}$, there exists a unique $a \in \mathbb{Z}/\left(n\mathbb{Z} \cap m\mathbb{Z}\right)$ such that: 
        \[\congruent{a}{a_1}{n}, \mathspace \congruent{a}{a_2}{m}\] 

        Now, an element of $\mathbb{Z}$ is equal to this $a$ if and only if it is of the form:
        \[\left\{a + \left(n\right) \cap \left(m\right)\right\} = \left\{a + nm\mathbb{Z}\right\}\]
        where we used that $\lcm\left(n, m\right) = \frac{nm}{\gcd\left(n, m\right)} = nm$.

        \qed
    \end{subparag}
\end{parag}

\begin{parag}{Definition: Pairwise coprime}
    Let $d_1, \ldots, d_r \in \mathbb{Z}$.

    We say that they are \important{pairwise coprime}, if $\gcd\left(d_i, d_j\right) = 1$ for any $i \neq j$.
\end{parag}


\begin{parag}{Theorem: Generalisation}
    Let $d_1, \ldots, d_r \in \mathbb{Z}$ be pairwise coprime.

    Then, for any $a_1, \ldots, a_r \in \mathbb{Z}$, there exists a $a \in \mathbb{Z}$ such that: 
    \[\begin{systemofequations} \congruent{a}{a_1}{d_1} \\ \vdots \\ \congruent{a}{a_r}{d_r}\end{systemofequations}\]
    
    The full set of solutions is: 
    \[\left\{a + \left(d_1 \cdots d_r\right)\mathbb{Z}\right\}\]
    
    \begin{subparag}{Proof}
        This proof can be done by induction. The base case was done by the previous corollary, and, for the inductive step, we suppose that we have a solution $a_k$ to the first $k$ equations, and we construct a $a_{k+1}$ which is a solution to the first $k+1$ equations.
    \end{subparag}
\end{parag}

\begin{parag}{Example}
    We want to find all solutions $a \in \mathbb{Z}$ of: 
    \[\begin{systemofequations} \congruent{a}{2}{5}\\ \congruent{a}{-1}{11}\\ \congruent{a}{3}{7}\\ \congruent{a}{0}{2} \end{systemofequations}\]
    
    Since $\left\{5, 11, 7, 2\right\}$ are pairwise coprime, we know that a solution exists by the CRT. We need to find a particular solution. A good way is just guesswork. We notice that $a = 10$ satisfies the last three equations: 
    \[\congruent{10}{3}{7}, \mathspace \congruent{10}{0}{2}, \mathspace \congruent{10}{-1}{11}\]
    
    By the CRT, We can therefore turn our system to the following equivalent one: 
    \[\begin{systemofequations} \congruent{a}{2}{5}\\ \congruent{a}{10}{154} \end{systemofequations}\]
    where $154 = \lcm\left(7, 2, 11\right) = 7\cdot 2\cdot 11$.

    Now, to solve this, we can use a direct method. A solution is of the form: 
    \[a = 154t + 10 = 5s + 2 \iff 154t - 5s = -8\]
    for some $t, s \in \mathbb{Z}$.

    We can use the extended Euclidean algorithm, or we can notice that 154 is one below a multiple of 5, i.e: 
    \[154\cdot 1 - 5\cdot 31 = -1 \implies 154\cdot 8 - 5\cdot \left(31\cdot 8\right) = -8 \implies 154\cdot 8 - 5\cdot 248 = -8\]
    
    We have thus got $t = 8$ and $s = 248$, telling us: 
    \[a = 154t + 10 = 154\cdot 8 + 10 = 1232 + 10 = 1242\]
    
    By the CRT, it tells us that the full set of solutions is: 
    \[\left\{1242 + 770\mathbb{Z}\right\}\]
    where $770 = 5\cdot 11 \cdot  7 \cdot 2$.

    In particular, this tells us that the smallest positive solution is $1242 - 770 = 472$.

    \begin{subparag}{Algorithmic method}
        Note that we can solve this kind of problem using a more algorithmic method. We find a solution $a_1$ such that: 
        \[\begin{systemofequations} \congruent{a_1}{1}{5}\\ \congruent{a_1}{0}{11}\\ \congruent{a_1}{0}{7}\\ \congruent{a_1}{0}{2} \end{systemofequations} \implies \begin{systemofequations} \congruent{a_1}{1}{5} \\ \congruent{a_1}{0}{11\cdot 7\cdot 2} \end{systemofequations}\]
        
        This can be easily done using the extended Euclidean algorithm. Indeed, $a_1$ must be of the form: 
        \[a_1 = 5s + 1 = \left(11\cdot 7\cdot 2\right)t \iff 5s - 154t = 1\]
        which can be indeed solved using this algorithm.
        
        Then, we find a solution $a_2$ such that:
        \[\begin{systemofequations} \congruent{a_2}{0}{5}\\ \congruent{a_2}{1}{11}\\ \congruent{a_2}{0}{7}\\ \congruent{a_2}{0}{2} \end{systemofequations}\]

        Continuing this way, we then just have: 
        \[a = 2 a_1 - a_2 + 3a_3 + 0a_4\]
    \end{subparag}

    \begin{subparag}{Remark}
        Explaining why a solution exists, describing the set of all solutions and finding the smallest positive solution is a typical exam question.
    \end{subparag}
\end{parag}

\begin{parag}{Definition: Group of units}
    Let $A$ be a ring. 

    Its \important{group of unit}, written $A^*$, is the subset of invertible elements with respect to multiplication.

    \begin{subparag}{Property 1}
        The group of unit is a multiplicative group.
    \end{subparag}

    \begin{subparag}{Property 2}
        Let $A, B$ be rings. Then: 
        \[\left(A \times B\right)^* \simeq A^* \times B^*\]
    \end{subparag}
\end{parag}

\begin{parag}{Proposition}
    Let $A, B$ be commutative rings.

    If $A \simeq B$ are isomorphic, then $A^* \simeq B^*$ are also isomorphic.
\end{parag}

\begin{parag}{Corollary}
    Let $n, m \in \mathbb{Z}$ such that $\gcd\left(n, m\right) = 1$.

    Then, $\phi\left(n m\right) = \phi\left(n\right)\phi\left(m\right)$.

    \begin{subparag}{Proof}
        By the CRT, we know that $\mathbb{Z}/mn\mathbb{Z} \simeq \mathbb{Z}/n\mathbb{Z} \times \mathbb{Z}/m\mathbb{Z}$, which means that: 
        \[\left(\mathbb{Z}/nm\mathbb{Z}\right)^* \simeq \left(\mathbb{Z}/n\mathbb{Z} \times \mathbb{Z}/m\mathbb{Z}\right)^* \simeq \left(\mathbb{Z}/n\mathbb{Z}\right)^* \times \left(\mathbb{Z}/m\mathbb{Z}\right)^*\]

        However: 
        \[\phi\left(nm\right) = \left|\left(\mathbb{Z}/nm\mathbb{Z}\right)^*\right| = \left|\left(\mathbb{Z}/n\mathbb{Z}\right)^*\right| \left|\left(\mathbb{Z}/m\mathbb{Z}\right)^*\right| = \phi\left(n\right)\phi\left(m\right)\]
        since isomorphic groups have the same number of elements.

        \qed
    \end{subparag}

    \begin{subparag}{Remark}
        Together with the fact that $\phi\left(p^a\right) = p^a - p^{a-1}$ for $p \in \mathbb{P}$ and $a \in \mathbb{N}^*$, it allows us to $\phi\left(n\right)$ for any $n$ for which we can find the prime factorisation.
    \end{subparag}
\end{parag}

\begin{parag}{Proposition: Converse of the CRT for integers}
    Let $n, m \in \mathbb{Z}$.

    If $\mathbb{Z}/nm\mathbb{Z} \simeq \mathbb{Z}/n\mathbb{Z} \times \mathbb{Z}/m\mathbb{Z}$, then $\gcd\left(n, m\right) = 1$.

    \begin{subparag}{Remark}
        This means that the converse of the CRT is true for integers. Note that this is not true for any field.
    \end{subparag}

    \begin{subparag}{Proof}
        We know by hypothesis that $\mathbb{Z}/nm\mathbb{Z} \simeq \mathbb{Z}/n\mathbb{Z} \times \mathbb{Z}/m\mathbb{Z}$. Therefore, their chacteristic must be equal: 
        \autoeq{nm = \tau\left(\mathbb{Z}/nm\mathbb{Z}\right) = \tau\left(\mathbb{Z}/n\mathbb{Z} \times \mathbb{Z}/m\mathbb{Z}\right) = \lcm\left(\tau\left(\mathbb{Z}/n\mathbb{Z}\right), \tau\left(\mathbb{Z}/m\mathbb{Z}\right)\right) = \lcm\left(n, m\right) = \frac{nm}{\gcd\left(n, m\right)}}
        
        However, this implies that $\gcd\left(n, m\right) = 1$, finishing our proof.

        \qed
    \end{subparag}
\end{parag}

\begin{parag}{Example}
    We know that, since $\gcd\left(16, 5\right) = 1$ and $16\cdot 5 = 80$, we have by the CRT that:
    \[\mathbb{Z}/80\mathbb{Z} \simeq \mathbb{Z}/16\mathbb{Z} \times \mathbb{Z}/5\mathbb{Z}\]
    
    However, even though $8\cdot 10 = 80$, we have by the converse of the CRT for integers that:
    \[\mathbb{Z}/80\mathbb{Z} \not\simeq \mathbb{Z}/8\mathbb{Z} \times \mathbb{Z}/10\mathbb{Z}\]
    since $\gcd\left(8, 10\right) = 2 \neq 1$.
\end{parag}

\subsection{Polynomial ring}

\begin{parag}{Definition: Polynomial ring}
    Let $A$ be a commutative ring. 

    The \important{ring of polynomials} over $A$ is:
    \[A \left[x\right] = \left\{a_0 + a_1 x + \ldots + a_n x^n | n \in \mathbb{N}, a_1, \ldots, a_n \in A\right\}\]
    with the usual addition and multiplication of polynomials.
\end{parag}

\begin{parag}{Definition: Degree}
    Let $f\left(x\right) \in A\left[x\right]$ be a polynomial.

    If it is non-zero, the \important{degree} of $f\left(x\right) = a_0 + \ldots + a_k x^k$ is defined as the largest $n \in \mathbb{N}$ such that $a_n \neq 0$. We note $\text{deg}\left(f\right) = n$.

    We also define the degree of the zero-polynomial to be $\text{deg}\left(0\right) = -\infty$.
    
    \begin{subparag}{Example}
        We for instance have: 
        \[\text{deg}\left(3\right) = 0, \mathspace \text{deg}\left(3x^2 - 5\right) = 2\]
    \end{subparag}
\end{parag}

\begin{parag}{Properties}
    Let $f, g \in A\left[x\right]$ be polynomials. Then:
    \begin{enumerate}
        \item $\deg\left(f + g\right) \leq \max\left(\deg f, \deg g\right)$
        \item If $A$ is an integral domain, $\deg\left(fg\right) = \deg f + \deg g$
    \end{enumerate}

    \begin{subparag}{Proof 1}
        We know that $\deg\left(f + g\right) = \max\left(\deg f, \deg g\right)$, unless $\deg f = \deg g$ and $a_n = -b_n$.

        For instance, summing two polynomials of degree 2, we can get a polynomial of degree 1:
        \[\left(3x^2 + 2x\right) + \left(-3x^2 + 5\right) = 2x + 5\]
    \end{subparag}
    
    \begin{subparag}{Proof 2}
        We notice that: 
        \autoeq{f\left(x\right)g\left(x\right) = \left(a_0 + a_1 x + \ldots + a_n x^n\right)\left(b_0 + b_1 x + \ldots + b_m x^m\right) = a_n b_m x^{n+m} + \text{lower degree terms}}
        
        Let's suppose that neither $f$ nor $g$ is $0$. Since the degree of $f$ is $n$ and the degree of $g$ is $m$, we know that that $a_n \neq 0$ and $a_m \neq 0$ by definition. Since $A$ is an integral domain, $a_n b_m \neq 0$, and thus $f\left(x\right)g\left(x\right)$ has degree $n + m$.

        Now, if $f$ or $g$ is the zero polynomial, then $fg = 0$. However, then, our property reads: 
        \[-\infty = -\infty + c\]
        for some $c \in \mathbb{N}_0 \cup \left\{-\infty\right\}$, which we take as a definition (which makes sense). This justifies the definition $\deg\left(0\right) = -\infty$.

        \qed
    \end{subparag}
\end{parag}

\begin{parag}{Proposition}
    Let $A$ be an integral domain.

    Then:
    \begin{enumerate}
        \item $A\left[x\right]$ is an integral domain.
        \item The units of $A\left[x\right]$ are the units of $A$.
    \end{enumerate}

    \begin{subparag}{Proof 1}
        Let's suppose that $f\left(x\right)g\left(x\right) = 0$. This means that: 
        \[\deg\left(fg\right) = -\infty\]

        However, we also know that $\deg\left(f\right) + \deg\left(g\right) = \deg\left(fg\right)$ since $A$ is an integral domain. If both $\deg\left(f\right)$ and $\deg\left(g\right)$ are finite, then their sum is also finite. This requires that either is $-\infty$, telling us that $f\left(x\right) = 0$ or $g\left(x\right) = 0$. This implies by definition that $A\left[x\right]$ is an integral domain.
    \end{subparag}

    \begin{subparag}{Proof 2}
        Let's suppose that $f\left(x\right)g\left(x\right) = 1$. This means that: 
        \[\deg\left(fg\right) = 0\]
        
        However, we also know that $\deg\left(f\right) + \deg\left(g\right) = \deg\left(fg\right)$ since $A$ is an integral domain. If any of the terms is $-\infty$, then the sum will also be $-\infty$. If either one is strictly positive, then their sum will also be strictly positive. This therefore means that they are both constant. In other words, $f\left(x\right) = a_0$ and $g\left(x\right) = b_0$, which are such that $a_0 \cdot b_0 = 1$ in $A$.

        \qed
    \end{subparag}
\end{parag}

\begin{parag}{Example}
    For instance, the following rings are integral domains: 
    \[\mathbb{R}\left[x\right], \mathspace \mathbb{Q}\left[x\right], \mathspace \mathbb{Z}\left[x\right]\]
    
    However, $\mathbb{Z}/6\mathbb{Z}\left[x\right]$ is not an integral domain. We for instance have that $\left(2x\right)\left(3x\right) = 0$ where $2x \neq 0$ and $3x \neq 0$.
\end{parag}

\begin{parag}{Theorem: Euclidean division in polynomial fields}
    Let $F$ be a field, and let $f\left(x\right), d\left(x\right) \in F\left[x\right]$ such that $\deg\left(d\right) \geq 1$.

    Then, there exists polynomials $q\left(x\right), r\left(x\right) \in F\left[x\right]$ be such that: 
    \[f\left(x\right) = q\left(x\right)d\left(x\right) + r\left(x\right)\]
    where $r\left(x\right) = 0$ or $\deg\left(r\right) < \deg\left(d\right)$.

    \begin{subparag}{Proof}
        Let's first suppose that $\deg\left(f\right) < \deg\left(g\right)$. Then, we can take: 
        \[f\left(x\right) = 0\cdot d\left(x\right) + f\left(x\right)\]
        
        Let's now suppose that $\deg\left(f\right) \geq \deg\left(g\right)$. The idea is that we can use the regular polynomial division algorithm. Let's describe it formally. We can write: 
        \[f\left(x\right) = a_0 + \ldots + a_m x^m\]
        \[d\left(x\right) = b_0 + \ldots + b_n x^n\]
        
        We now construct the following polynomial:
        \[p_1\left(x\right) = f\left(x\right) - d\left(x\right) \cdot \frac{a_m}{b_n} x^{m-n}\]
        where $\frac{a_m}{b_n} = a_m b_n^{-1} \in F$ since it is a field. 

        If $\deg\left(p_1\right) < \deg\left(f\right)$, we are done. Otherwise, we can can repeat: 
        \autoeq{p_2\left(x\right) = p_1\left(x\right) - d\left(x\right)\frac{a_{m-1}}{b_n}x^{m - n -1} = f\left(x\right) - d\left(x\right)\frac{a_m}{b_n}x^{m-n} - d\left(x\right)\frac{a_{m-1}}{b_n}x^{m-n-1}}
        
        The sequence of degrees is strictly decreasing, so, at some point, the process terminates with a $p\left(x\right) = f\left(x\right) - d\left(x\right) q\left(x\right)$ such that: 
        \[f\left(x\right) -d\left(x\right)q\left(x\right) = r\left(x\right)\]

        \qed
    \end{subparag}

    \begin{subparag}{Remark}
        As mentioned above, this is just the regular polynomial division algorithm. For instance, if we have $f\left(x\right) = 3x^5 + x^3 - 2x^2 + 1$ and $d\left(x\right) = x^2 - 2$, we have: 
        \imagehere[0.6]{PolynomialDivisionExample.png}
        where, for example, the first term of $q\left(x\right)$ was chosen to be $3x^3$ since $3x^3\left(x^2 - 2\right) = 3x^5 + \ldots$, which allows to decrease the degree of $f\left(x\right)$.

        This allows us to write:
        \autoeq{f\left(x\right) = d\left(x\right)q\left(x\right) + r\left(x\right) \iff 3 x^5 + x^3 - 2x^2 + 1 = \left(x^2 - 2\right)\left(3x^3 + 7x - 2\right) + \left(14x - 3\right)}
    \end{subparag}
\end{parag}

\subsection{Euclidean domains}

\begin{parag}{Definition: Euclidean domain}
    Let $A$ be a commutative ring. 

    It is said to be a \important{Euclidean domain} if:
    \begin{enumerate}
        \item $A$ is an integral domain.
        \item There exists a function $\nu: A \setminus \left\{0\right\} \mapsto \mathbb{N}$ such that for any $a, b \in A$ where $b \neq 0$, there exists $q, r \in A$ such that: 
        \[a = qb + r\]
        and either $r = 0$ or $\nu\left(r\right) < \nu\left(b\right)$.
    \end{enumerate}

    \begin{subparag}{Intuition}
        This generalises abstractly any space which admits a Euclidean division.
    \end{subparag}

    \begin{subparag}{Example}
        We can for instance consider the following Euclidean domain, which is the main one we have been worked with since the beginning of the course:
        \[\mathbb{Z}, \mathspace \nu\left(n\right) = \left|n\right|\]

        It is also possible (though harder than exam material) to show that the following integral domain is a Euclidean domain: 
        \[\mathbb{Z}\left[i\right] = \left\{a + ib \suchthat a, b \in \mathbb{Z}\right\}, \mathspace \nu\left(a + ib\right) = \left|a + ib\right|^2 = a^2 + b^2\]
        
    \end{subparag}
\end{parag}

\begin{parag}{Proposition}
    Let $A$ be a field.

    Then, $A$ is a Euclidean domain.

    \begin{subparag}{Proof}
        For any $a, b \in A$, we can find a $q \in A$ such that: 
        \[a = qb + 0 = qb\]
        
        We can therefore take $\nu$ to be any function.

        \qed
    \end{subparag}
\end{parag}

\begin{parag}{Corollary}
    Let $\mathbb{F}$ be a field.

    Then, $\mathbb{F}\left[x\right]$ is a Euclidean domain.

    \begin{subparag}{Proof}
        We can just take $\nu = \deg$. 

        Then, given $f\left(x\right), d\left(x\right) \in \mathbb{F}\left[x\right]$, we know that we can find $q\left(x\right), r\left(x\right)$ such that
        \[f\left(x\right) = q\left(x\right)d\left(x\right) + r\left(x\right)\]
        where $r\left(x\right) = 0$ or $\deg\left(r\right) < \deg\left(d\right)$.

        \qed
    \end{subparag}
\end{parag}

\begin{parag}{Proposition}
    Let $A$ be a Euclidean domain.

    Then, $A$ is a PID (principal ideal domain).

    \begin{subparag}{Proof}
        Let $E$ be a Euclidean domain of function $\nu$, and $I \subset E$ be an ideal.

        If $I = \left\{0\right\}$, then we directly have $I = \left(0\right)$.

        Let's now suppose that $I \neq \left\{0\right\}$. We can pick a $d \in I \setminus \left\{0\right\}$ such that $\nu\left(d\right)$ is the minimum on $I$. Now, let $a \in I$ be arbitrary. This yields that there exists $q, r$ such that $\nu\left(r\right) < \nu\left(d\right)$ or $r = 0$, and: 
        \[a = qd + r \implies r = \underbrace{a}_{\in I} - \underbrace{qd}_{\in I} \in I\]

        However, since $r \in I$ and we picked $d$ such that $\nu\left(d\right)$ was the minimum on $I$, we cannot have $\nu\left(r\right) <\nu\left(d\right)$. This means that $r = 0$, and thus that: 
        \[a = qd \implies I = \left(d\right)\]
        
        Any ideal is principal, showing this is indeed a PID.
        
        \qed
    \end{subparag}

    \begin{subparag}{Remark}
        This for instance implies that, for a field $F$, $F\left[x\right]$ is a PID, i.e. all its ideals are generated by a single element.
    \end{subparag}
\end{parag}

\begin{parag}{Remark}
    We have proven the following inclusions: 
    \[\text{Fields} \subset \text{Euclidean domains} \subset \text{PID} \subset \text{Integral domain} \subset \text{Commutative rings}\]
    
    We have the following examples:
    \begin{enumerate}
        \item Fields: 
        \[\mathbb{R}, \mathspace \mathbb{C}, \mathspace \mathbb{Z}/p\mathbb{Z}\]
        for $p \in \mathbb{P}$ prime.
        \item Euclidean domains that are not fields: 
        \[\mathbb{Z}, \mathspace \mathbb{R}\left[x\right]\]
        \item PID that are not Euclidean domain: \textit{``They exist, but don't worry about it.''}
        \item Integral domains that are not PIDs:
        \[\mathbb{R}\left[x, y\right]\]
        \item Commutative rings that are not integral domains: 
        \[\mathbb{Z}/n\mathbb{Z}\]
        for $n \in \mathbb{N}_{\geq 2} \setminus \mathbb{P}$ composite.
    \end{enumerate}
\end{parag}

\end{document}
