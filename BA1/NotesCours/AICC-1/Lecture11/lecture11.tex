\documentclass[a4paper]{article}

% Expanded on 2021-10-26 at 08:30:21.

\usepackage{../../style}

\title{AICC 1}
\author{Joachim Favre}
\date{Mardi 26 octobre 2021}

\begin{document}
\maketitle

\lecture{11}{2021-10-26}{Introduction to algorithms}{
\begin{itemize}[left=0pt]
    \item History and definition of algorithms.
    \item Explanation of two searching algorithms: linear and binary search.
    \item Explanation of two sorting algorithms: bubble sort and insertion sort.
\end{itemize}

}

\parag{Revision to countability}{
    The set $\left[0,1\right] $ of real numbers is uncountable.

    \subparag{Proof}{
        We will use binary representation for more fun (for example: $\left(0.1\right)_2 = 0.5$ and $\left(0.11\right)_2 = 0.75$).

        Let's suppose for contradiction that this set is countable. Thus, it can be written as a sequence (it does not matter how we write it, it must always be possible):
    \begin{center}
    \begin{tabular}{rcl}
        $r_0$ & $=$ & $ 0.{\color{red}1}010101\ldots$ \\
        $r_1$ & $=$ & $ 0.0{\color{red}0}10010\ldots$ \\
        $r_2$ & $=$ & $ 0.11{\color{red}0}0110\ldots$ \\
        & $\vdots$ & \\
        $r_i$ & $=$ & $ 0.d_{i1}d_{i2}\ldots {\color{red}d_{ii}} \ldots$  \\
        & $\vdots$ & \\
    \end{tabular}
    \end{center}

    Now we can construct $r = 0.011\ldots$ with every bit on the diagonal (in red), which we invert ($\bar{d_i} = 1 - d_{ii}$). Then it cannot be in the list, since the bit on the diagonal part would not be equal (we would have inverted it). Since it is not in the list, it is a contradiction to the fact that we can construct such a list.

    \qed
    }

    \subparag{Remark}{
        We can use the same argument to show that the power set of the natural numbers is uncountable. We can use a number as each bit saying ``element $i$ is in the power set''.
    }


}

\section{Algorithms}
\subsection{Introduction}
\parag{Definition}{
    An algorithm is a finite set of well-defined instructions to perform a specified task. Meaning, to perform a computation, to solve a problem, to reach a certain destination, and so on.

    Note that we can do algorithms without computers.
}

\parag{History}{
    There are algorithms ranging from 2500 BC in Babylon, that allowed to perform a division, Greece in 300 BC with finding prime number, from Muhammed ibn Musa al-Khwarizmi (the best, his name gave ``algorithm'') in 780 BC with solving quadratic equations. Then, in the twentieth century, many people such as Hilbert, Gödel, Church, and Kleene worked on predicate logic and, Turing came up with the Turing Machine, which is the basis for algorithms on computers.
}

\parag{Example}{
    Let's suppose we want to find the maximum value in a finite sequence of integers. Then we can use the following algorithm:
   \begin{enumerate}
       \item Set the temporary maximum equal to the first integer in the sequence.
       \item Compare the next integer in the sequence to the temporary maximum. If it is larger than the temporary maximum, then we set the latter to this integer.
       \item When we get at the end of the list, our temporary maximum is the global maximum.
   \end{enumerate}
}

\begin{filecontents*}[overwrite]{max.code}
procedure max(a: array of n integers):
    tmp_max := a[1]
    for i := 2 to n:
        if tmp_max < a[i] then tmp_max := a[i]
    return tmp_max
\end{filecontents*}


\parag{Specifying algorithms}{
    We have multiple ways to describe algorithms: natural language, pseudo-code and programming languages.

    Pseudo-code is an interemediate step, and it is used very often because it is more precise than natural language, but more general than a specfifc programming language. The idea is that if we know any typical programming language, then the syntax is clear enough to understand. For example:
    \importcode{max.code}{pseudo}
    
}

\parag{Typical problems}{
   \begin{description}[left=0pt]
       \item[Searching problems:] Find an element in a list.
       \item[Sorting problems:] Put elements in a list in order.
       \item[Optimisation problems:] Determine the optimal value of a particular quantity over all possible inputs.
   \end{description}

   The two first ones are really important for managing data.

}

\subsection{Searching algorithms}
\parag{Searching problem}{
    Given a list $S = a_1, \ldots, a_n$ of distinct elements, and some $x$. If $x \in S$, then it must return $i$ such that $a_i = x$, else it must return 0.
}

\begin{filecontents*}[overwrite]{linearsearch.code}
procedure linear_search(x: integer, a: array of n distinct integers):
    i := 1
    while(i <= n and x != a[i])
        i := i + 1
    if i == n + 1 then location := 0 else location := i
    return location
\end{filecontents*}

\parag{Linear search algorithm}{
    The most simple one is to traverse the list, considering elements one by one, starting at the beginning. If we reach the element, then we have found it, else if we get out of the list without finding it, then it is not in it. Here is the pseudocode:
    \importcode{linearsearch.code}{pseudo}
}

\parag{Example}{
    Let's say we are looking for $x = 2$:
    \begin{center}
    \begin{tabular}{r|cccccc}
        sequence & 3 & 5 & 1 & 7 & 2 & 1 \\
        \hline
        $x \neq a_i$ & T & T & T & T & F & \\
        location & 2 & 3 & 4 & 5 &
    \end{tabular}
    \end{center}

}

\begin{filecontents*}[overwrite]{binarysearch.code}
procedure binary_search(x: integer, a: array of n increasing integers)
    i := 1  // left-most index
    j := n  // right-most index
    while i < j
        m := floor((i + j)/2)
        if x > a[m] then i := m+1 else j := m
    if x = a[i] then location := i else location := 0
    return location
\end{filecontents*}


\parag{Binary Search}{
    Let's assume that, this time, the input is a list of items in increasing order (they have been sorted previously).

    \begin{enumerate}
        \item We begin by comparing the element to be found with the middle element.
            \begin{enumerate}
                \item If the middle element is lower, the search proceeds with the upper half of the list.
                \item If it is not lower, the search with the lower part of the list (including the middle position).
            \end{enumerate}
        \item Repeat this process until we have a list of size $1$.
            \begin{enumerate}
                \item If the element we are looking for is equal to the element in the list, the position is returned.
                \item Otherwise, 0 is returned to indicate that the element was not found.
            \end{enumerate}
    \end{enumerate}

    \importcode{binarysearch.code}{pseudo}
}

\parag{Example}{
    Let's say we are looking for 19 in the following list of 16 elements:
    \[1, 2, 3, 5, 6, 7, 8, 10{\color{red},} \underline{12, 13, 15, 16, 18, 19, 20, 22}\]

    Cutting in halves:
    \[12, 13, 15, 16{\color{red},} \underline{18, 19, 20, 22}\]

    Again:
    \[18, 19{\color{red},} \underline{20, 22}\]
    \[18{\color{red},} \underline{19}\]
    \[19\]

    We have a list of only one element, and it contains 19. So, we have found it.
}

\subsection{Sorting algorithms}
\parag{Sorting problems}{
    Given a list $S = a_1, \ldots, a_n$, we want to return a list where the elements are put in increasing order.
}

\begin{filecontents*}[overwrite]{bubblesort.code}
procedure bubble_sort(a : array of n >= 2 real numbers):
    for i := 1 to n - 1
        for j := 1 to n - i
            if a[j] > a[j+1] then interchange(a[j], a[j+1])  
\end{filecontents*}


\parag{Bubble sort}{
    We will do different passes through the list:
    \begin{enumerate}
        \item In one pass, every pair of elements that are found to be out of order are interchanged.
        \item Since the last element is guaranteed to be the largest after the first pass, in the second pass it does not need anymore to be inspected.
        \item We can repeat this process, and every pass we have one less element to inspect.
    \end{enumerate}

    \importcode{bubblesort.code}{pseudo}
    
}

\parag{Example}{
    Let's say we have the list
    \[3, 2, 4, 1, 5\]

    Then, comparing the first two elements, we realise that $3 < 2$ is wrong, so:
    \[2, 3, 4, 1, 5\]

    Now $3 < 4$ is good. However, $4 < 1$ is not, so we end up with:
    \[2, 3, 1, 4, 5\]

    We can finish our first pass since $4 < 5$. We indeed have that the greatest element, $5$, is at the end of the list. We can repeat this process on the sub-list $2, 3, 1, 4, 5$ and we will end up with :
    \[1, 2, 3, 4, 5\]
    as expected.

    Animations of this algorithms are really the best way to understand it.

    \begin{center}
    \begin{tabular}{cccc|ccc|cc|c|c}
        \textcolor{red}{3} & 2 & 2 & 2 & \textcolor{red}{2} & 2 & 2 & \textcolor{red}{2} & 1 & \textcolor{red}{1} & \textcolor{maincolour}{1}\\
        \textcolor{red}{2} & \textcolor{red}{3} & 3 & 3 & \textcolor{red}{3} & \textcolor{red}{3} & 1 & \textcolor{red}{1} & \textcolor{red}{2} & \textcolor{red}{2} & \textcolor{maincolour}{2} \\
        4 & \textcolor{red}{4} & \textcolor{red}{4} & 1 & 1 & \textcolor{red}{1} & \textcolor{red}{3} & 3 & \textcolor{red}{3} & \textcolor{maincolour}{3} & \textcolor{maincolour}{3} \\
        1 & 1 & \textcolor{red}{1} & \textcolor{red}{4} & 4 & 4 & \textcolor{red}{4} & \textcolor{maincolour}{4} & \textcolor{maincolour}{4} & \textcolor{maincolour}{4} & \textcolor{maincolour}{4}  \\
        5 & 5 & 5 & \textcolor{red}{5} & \textcolor{maincolour}{5} & \textcolor{maincolour}{5} & \textcolor{maincolour}{5} & \textcolor{maincolour}{5} & \textcolor{maincolour}{5} & \textcolor{maincolour}{5} & \textcolor{maincolour}{5} \\
        \hline
        \multicolumn{4}{c}{Pass 1} & \multicolumn{3}{c}{Pass 2} & \multicolumn{2}{c}{Pass 3} & \multicolumn{1}{c}{Pass 4} & End
    \end{tabular}
    \end{center}

}

\begin{filecontents*}[overwrite]{insertionsort.code}
procedure insertion_sort(a: array of n >= 2 real numbers):
    for j := 2 to n
        i := 1
        while a[j] > a[i] and i < j  // move a[j] to the right position
            i := i + 1
        m := a[j]
        for k := 0 to j - i - 1  // shift elements to make space for m = a[j]
            a[j-k] := a[j-k-1]
        a[i] := m
\end{filecontents*}

\parag{Insertion sort}{
    \begin{enumerate}[left=0pt]
        \item Compare the second element with the first and puts it before the first if it is not larger.
        \item Then the third element is put in corret position among the three first using linear search.
        \item We continue this way every pass, increasing the size of our sorted subarray.
    \end{enumerate}

    \importcode{insertionsort.code}{pseudo}
    
}

\parag{Example}{
    \begin{tabular}{ccccc}
        3 & 2 & 2 & 1 & \textcolor{maincolour}{1}\\
        \textcolor{red}{2} & 3 & 3 & 2 & \textcolor{maincolour}{2}\\
        4 & \textcolor{red}{4} & 4 & 3 & \textcolor{maincolour}{3}\\
        1 & 1 & \textcolor{red}{1} & 4 & \textcolor{maincolour}{4}\\
        5 & 5 & 5 & \textcolor{red}{5} & \textcolor{maincolour}{5}
    \end{tabular}

}

\end{document}
