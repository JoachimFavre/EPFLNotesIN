\documentclass[a4paper]{article}

% Expanded on 2021-10-19 at 08:39:04.

\usepackage{../../style}

\title{AICC-1}
\author{Joachim Favre}
\date{Mardi 19 octobre 2021}

\begin{document}
\maketitle

\lecture{9}{2021-10-19}{Equivalence relations and partial orderings}{
\begin{itemize}[left=0pt]
    \item Definition of equivalence relations
    \item Definition of equivalence classes, and explanation of their equivalence to set partitions.
    \item Definition of partial orderings (poset), and Hasse Diagrams.
    \item Definition of total ordered and well-ordered sets.
\end{itemize}

}

\subsection{Equivalence relations}

\parag{Definition: Equivalence}{
    A relation on a set $A$ is called an \important{equivalence relation} if it is reflexive, symmetric, and transitive.

    Two elements $a$ and $b$ are related by an equivalence relation are called \important{equivalent}, often written $a \sim b$.

    \subparag{Motivation}{
        Let's suppose we have coins of different denominations (4 times 1, 3 times 2 and 2 times 5, for example). Then we may want to make a relation $R$ which says that ``coin $x$ has the same value as coin $y$'', because they are different coins but to a shopper it would make no difference to pay with one two francs coin or another of the same value.

    We realise that $R$, in this cas, is symmetric, reflexive and transitive. We generalise equivalence relations using those properties.

    }
}

\parag{Example 1}{
    Let the following relation:
    \[R_{minus} = \left\{\left(a, b\right) \in \mathbb{R} \times \mathbb{R} |\ a - b \in \mathbb{Z}\right\}\]

    We can see that two numbers ($a = 3.142$ and $b = 2.142$, for example) are in the relation if and only if they have the same decimal digits.

    \begin{enumerate}[left=0pt]
        \item We see that it is reflexive since
            \[a - a = 0 \in \mathbb{Z}\]
        \item It is symmetric since
            \[\left(a-b\right) \in \mathbb{Z} \implies \left(b - a\right) = -\underbrace{\left(a - b\right)}_{\in \mathbb{Z}} \in \mathbb{Z}\]
        \item It is transitive since
        \[\left(a - b\right), \left(b - c\right) \in \mathbb{Z} \implies a - c = \underbrace{\left(a-b\right)}_{\in \mathbb{Z}} + \underbrace{\left(b - c\right)}_{\in \mathbb{Z}} \in \mathbb{Z}\]
    \end{enumerate}

    So, it is an equivalence.
}

\parag{Example 2}{
    Let the following relation:
    \[R_{divides} = \left\{\left(a, b\right) \in \mathbb{N} |\ a \text{ divides } b\right\}\]

    We notice that it is not an equivalence, since it is not symmetric. For example, $2 \divides 4$ but $4 \ndivides 2$.
}

\parag{Definition: Equivalence class}{
    Let $R$ be an equivalence relation on a set $A$. The set of all elements that are related to an element $a$ of $A$ is called the \important{equivalence class} of $a$, denoted $\left[a\right]_{R}$, or $\left[a\right]$.

    In other words:
    \[\left[a\right]_R = \left\{s \in A |\ \left(a, s\right) \in R\right\}\]

    If $b \in \left[a\right]_R$, then $b$ is called a \important{representative} of this equivalence class.
}

\parag{Example}{
    Let's come back to our previous equivalence relation:
    \[R_{minus} = \left\{\left(a, b\right) \in \mathbb{R} \times \mathbb{R} |\ a - b \in \mathbb{Z}\right\}\]

    We have the following equivalence classes:
    \[\left[0\right]_{\mathcal{R}_{\text{minus}}} = \mathbb{Z}\]
    \[\left[3.142\right]_{\mathcal{R}_{\text{minus}}} = \left\{3.142, 4.142, 2.142, \ldots\right\}\]
}

\parag{Theorem}{
    Let $R$ be an equivalence relation on a set $A$. These statements for elements $a$ and $b$ of $A$ are equivalent:
    \begin{enumerate}
        \item $R\left(a, b\right)$
        \item $\left[a\right] = \left[b\right]$
        \item $\left[a\right] \cap \left[b\right] \neq \o$
    \end{enumerate}

}

\parag{Definition: Partition of a set}{
    A \important{partition} of a set is a collection of disjoint nonempty subsets of $S$ that have $S$ as their union. More formally, the collection of subsets $A_i \subset S$ form a partiition of $S$ if and only if:
    \[A_i \neq \o, \mathspace i \neq j \implies A_i \cap A_j = \o, \mathspace \bigcup_{i} A_i = S\]

    \subparag{Illustration}{
        \imagehere[0.7]{setPartitionIllustration.png}
    }
}

\parag{Theorem}{
    Let $R$ be an equivalence relation on a set $S$.

    Then, the equivalence classes of $R$ form a partition of $S$. Conversely, given a partition $\left\{A_i\right\}$ of the set $S$, there is an equivalence relation $R$ that has the sets $A_i$ as its equivalence classes.

    \subparag{Other formulation}{
        Set partitions and equivalence classes are equivalent: we can construct one from another.
    }

    \subparag{Illustration}{
        Let's take back the example we did to introduce this subject, i.e. two coins are in an equivalence relation if and only if they have the same denomination. We can make the following drawings:
        \imagehere{equivalenceSetPartitionEquivalenceRelation.png}

        On the left, we see the equivalence relation, on the right the set partition.
    }
}

\subsection{Partial orderings}
\parag{Definition}{
    A relation $R$ on a set $S$ is called a \important{partial ordering}, or \important{partial order}, if it is reflexive, antisymmetric, and transitive

    A set together with a partial ordering $R$ is called a \important{partially ordered set}, or \important{poset}, and is denoted by $\left(S, R\right)$.

    \subparag{Motivation}{
        Let's say we still have our coins, but we want to sort them. Let's pick $R$ to be the relation ``coin $x$ has at least the value of coin $y$''.

        We can see that $\mathcal{R}$ is transitive, reflexive and anti-symmetric. This is a \important{total order}.

        Let's say we want to compare purses. Let's pick $\mathcal{R}$ be the relation ``purse $x$ has at least as many coins of any value as purse $y$''. If we have $\left(a, b\right)$, a purse of $a$ two francs coins and $b$ one francs coin, then we have $R\left(\left(2,1\right), \left(1, 1\right)\right)$ and $\mathcal{R\left(\left(1,2\right), \left(1,1\right)\right)}$. However, there is no relation between $\left(1, 2\right)$ and $\left(2, 1\right)$, we cannot compare them.  This is called a \important{partial ordering}.
    }
}

\parag{Definition: Comparability}{
    The elements $a$ and $b$ of a poset $\left(S, \preccurlyeq\right)$ are \important{comparable} if either $a \preccurlyeq b$ or $b \preccurlyeq a$. When $a$ and $b$ are elements of $S$ so that neither $a \preccurlyeq b$ nor $b \preccurlyeq a$, then $a$ and $b$ are called \important{incomparable}.

    \subparag{Remark}{
        Note that the symbol $\preccurlyeq$ is often used to denote the relation in posets. This is done as an analogy to $\leq$, since $\left(S, \leq\right)$ is a poset for any set $S$.
    }
}

\parag{Example 1}{
    We can see that $\left(\mathbb{Z}, \geq\right)$ is a poset.

    Indeed, it is easy to show that it is reflexive, since $a \geq a$ is true for all integers. It is clear that $a\geq b$ and $b \geq a$ implies $a = b$, so it is anti-symmetric. Finally, it is transitive, since $a \geq b$ and $b \geq c$ implies that $a \geq c$.
}

\parag{Example 2}{
    We can see that $\left(\mathbb{Z}_+, \divides\right)$ is a poset:
    \begin{description}
        \item[Reflexive:] $a \divides a$ is true for all integers.
        \item[Anti-symmetric:] Let's say we have $a \divides b$ and $b \divides a$. In other words, we have $a = k_1 b$ and $b = k_2 a$. Thus:
            \[a = k_1 b = k_1 k_2 a \implies k_1 k_2 = 1\]
            Since $k_1, k_2 \in \mathbb{Z}$, we have $k_1 = k_2 = 1$, and therefore $a = b$.
        \item[Transitive:] Let's say we have $c \divides b$ and $b \divides a$, i.e. $c = k_1 b$ and $b = k_2 a$. So:
            \[c = k_1b = k_1 k_2 a \implies c \divides a\]
    \end{description}
}

\parag{Example 3}{
    We can see that $\left(\mathcal{P}\left(S\right), \subseteq\right)$ is a poset:
    \begin{description}
        \item[Reflexive:] $A \subseteq A$, indeed.
        \item[Anti-symmetric:] We know by definition of set equality that:
        \[A \subseteq B, B \subseteq A \implies A = B\]
        \item[Transitive:] We know that:
        \[A \subseteq B, B \subseteq C \implies A \subseteq C\]
    \end{description}
}

\parag{Example 4}{
    The following are not posets, since they are not reflexive:
    \[\left(\mathbb{Z}, \neq\right), \mathspace \left(\mathbb{Z}, \ndivides\right), \mathspace \left(\mathbb{R}, <\right)\]
}

\parag{Hasse diagrams}{
    We know that all partial orderings are reflexive, transitive and anti-symmetric.

    We can draw a directed graph (a). Then, we can omit self-loops (since they are all reflexive) (b). Finally, we can remove transitive edges, and assume that arrows point upwards (no arrow point in both direction since relations are anti-symmetric). The last diagram is a Hasse diagram.

    \imagehere[0.4]{HasseDiagramConstruction.png}
}

\parag{Example}{
    Let's draw the Hasse diagram of $\left(\mathcal{P}\left(\left\{a, b, c\right\}\right), \subseteq\right)$:
    \imagehere[0.4]{HasseDiagramExample.png}

    \subparag{Personal note}{
        Fun fact, this draws exactly a 3D cube.
    }

}

\parag{Definition}{
    If $\left(S, \preccurlyeq\right)$ is a poset and every two elements of $S$ are comparable, then $S$ is called a \important{totally ordered} or \important{linearly ordered set}, and $\preccurlyeq$ is called a \important{total order} or a \important{linear order}.

    We say that $\left(S, \preccurlyeq\right)$ is a \important{well-ordered} set if it is a poset such that $\preccurlyeq$ is a total ordering, and every nonempty subset of $S$ has a least element.

    \subparag{Note}{
        One of the defining axiom of the natural numbers, is its well-ordering, i.e that $\left(\mathbb{N}, \leq\right)$ is well-ordered.
    }

    \subparag{Example}{
        $\left(\mathbb{Z}, \leq\right)$ is totally ordered, and $\left(\mathbb{Z}_+, \divides\right)$ and $\left(\mathcal{P}\left(S\right), \subseteq\right)$ (with $\left|S\right| > 1$) are not totally ordered.
    }

    \subparag{Hasse Diagrams}{
        We can draw the following four Hasse diagrams.
        \imagehere[0.5]{HasseDiagramsTotallyOrderedSets.png}

        They are all Hasse diagrams of totally ordered sets, and the two first ones are well ordered.
    }
}

\parag{Definition}{
    Let $\left(S, \preccurlyeq\right)$ be a partially ordered set.

    An \important{upper bound} $u \in S$ of $A \subseteq S$ is an element such that:
    \[a \preccurlyeq u, \mathspace \forall a \in A\]

    A \important{lower bound} $u \in S$ of $A \subseteq S$ is an element such that:
    \[u \preccurlyeq a, \mathspace \forall a \in A\]

    \subparag{Remark}{
        Note that $u$ is an element of $S$, thus it is not necessarily element of $A$.

        For example, using $\left(\mathcal{P}\left(\left\{a, b, c\right\}\right), \subseteq\right)$ (its Hasse diagram can be found hereinabove), if we have $A = \left\{\left\{a\right\}, \left\{b\right\}, \left\{c\right\}\right\}$, then the only possible upper bound is $\left\{a,b,c\right\}$.
    }

    A \important{least upper bound} $u \in S$ of $A \subseteq S$ is an upper bound of $A$ such that it is less than every other upper bound of $A$.

    A \important{greatest  lower bound} $u \in S$ of $A \subseteq S$ is a lower bound of $A$ such that it is greater than every other bound of $A$.

    \subparag{Remark}{
        The least upper bound and greatest lower bound of a subset $A$ are unique, if they exist (this follows directly from anti-symmetry). Note that they do not always exist (if we have bounds that are not comparables, for instance).
    }
}

\end{document}
