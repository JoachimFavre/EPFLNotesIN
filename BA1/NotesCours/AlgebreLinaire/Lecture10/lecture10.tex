\documentclass[a4paper]{article}

% Expanded on 2021-10-25 at 13:15:31.

\usepackage{../../style}

\title{Algèbre linéaire}
\author{Joachim Favre}
\date{Lundi 25 octobre 2021}

\begin{document}
\maketitle

\lecture{10}{2021-10-25}{Entrée dans l'hyper-espace}{
\begin{itemize}[left=0pt]
    \item Définition des espaces vectoriels, et des propriétés qu'ils doivent suivre.
    \item Définition des sous-espaces vectoriels, et des propriétés qu'ils doivent suivre.
\end{itemize}

}

\section{Espaces et sous-espaces vectoriels}
\subsection{Espaces vectoriels}

\parag{Introduction}{
    Avec ce chapitre, nous faisons un bon en abstraction. Cela demande un effort intellectuel important. Cet effort sera récompensé parce qu'il va rendre nos outils mathématiques bien plus largement applicables.

    Jusqu'ici nous avons travaillé dans
    \[\mathbb{R}^{n} = \left\{\begin{bmatrix} x_1 \\ \vdots \\ x_n \end{bmatrix} \telque x_1, \ldots, x_n \in \mathbb{R}\right\}\]

    On a appelé $\bvec{x} \in \mathbb{R}^{n}$ un vecteur.

    Sans y prêter attention, nous avons utilisé les propriétés suivantes de $\mathbb{R}^{n}$, valides pour tout $\bvec{u}, \bvec{v}, \bvec{w} \in \mathbb{R}^{n}$ et $c, d \in \mathbb{R}$:

    \begin{center}
    \begin{minipage}{0.45\textwidth}
        \begin{itemize}[left=0pt]
        \item $\bvec{u} + \bvec{v} \in \mathbb{R}^{n}$
        \item $\bvec{u} + \bvec{v} = \bvec{v} + \bvec{u}$
        \item $\left(\bvec{u} + \bvec{v}\right) + \bvec{w} = \bvec{u} + \left(\bvec{v} + \bvec{w}\right)$
        \item $\bvec{u} + \bvec{0} = \bvec{u}$
        \item $\bvec{u} + \left(-\bvec{u}\right) = \bvec{0}$
    \end{itemize}
    \end{minipage}
    \hfill
    \begin{minipage}{0.45\textwidth}
    \begin{itemize}[left=0pt]
        \item $c \bvec{u} \in\mathbb{R}^{n}$
        \item $c\left(\bvec{u} + \bvec{v}\right) = c \bvec{u} + c \bvec{v}$
        \item $\left(c + d\right)\bvec{u} = c \bvec{u} + d \bvec{u}$
        \item $c\left(d \bvec{u}\right) = \left(cd\right)\bvec{u}$
        \item $1 \bvec{u}= \bvec{u}$
    \end{itemize}
    \end{minipage}
    \end{center}
}

\parag{Généralisation}{
    En réalité, ce sont là \textit{toutes} les propriétés de $\mathbb{R}^{n}$ que nous avons utilisées. Donc, si au lieu de $\mathbb{R}^{n}$ on a un autre ensemble $V$ muni d'opérations ``d'addition de deux éléments'' et de ``multiplication par un scalaire'', et que ces opérations ont toutes les propriétés ci-dessus, alors on peut refaire tout notre travail des semaines précédentes dans $V$ plutôt que dans $\mathbb{R}^{n}$.

    On aura les notions de combinaison linéaire, d'indépendance linéaire, d'espace engendré (``vect''), d'application linéaire, etc.

    On appelle un ensemble $V$ muni de telles opérations un \important{espace vectoriel}. Un \important{vecteur} est un élément $\bvec{v}$ dans $V$.
}

\parag{Définition}{
    On appelle un \important{espace vectoriel} tout ensemble non vide $V$ constitué d'objets appelés vecteurs, sur lequel sont définies deux opérations appelées addition et multiplication par un scalaire (nombre réel). Ces opérations vérifient les dix axiomes énumérées ci-après, quels que soient les vecteurs $\bvec{u}$, $\bvec{v}$ et $\bvec{w}$ de $V$ et les scalaires $c$ et $d$:
    \begin{enumerate}
        \item La somme de $\bvec{u}$ et $\bvec{v}$, notée $\bvec{u} + \bvec{v}$, est dans $V$.
        \item $\bvec{u} + \bvec{v} = \bvec{v} + \bvec{u}$
        \item $\left(\bvec{u} + \bvec{v}\right) + \bvec{w} = \bvec{u} + \left(\bvec{v} + \bvec{w}\right)$
        \item Il existe un vecteur de $V$ dit vecteur nul, ou zéro, noté $\bvec{0}$, tel que $\bvec{u} + \bvec{0} = \bvec{u}$
        \item Pour tout vecteur de $\bvec{u}$ de $V$, il existe un vecteur $-\bvec{u}$ de $V$ tel que $\bvec{u} + \left(-\bvec{u}\right) = \bvec{0}$.
        \item Le produit du vecteur $\bvec{u}$ par le scalaire $c$, noté $c \bvec{u}$, est dans $V$.
        \item $c\left(\bvec{u} + \bvec{v}\right) = c \bvec{u} + c \bvec{v}$
        \item $\left(c + d\right)\bvec{u} = c \bvec{u} + d \bvec{u}$
        \item $c\left(d \bvec{u}\right) = \left(cd\right)\bvec{u}$
        \item $1 \bvec{u} = \bvec{u}$
    \end{enumerate}

    En d'autres mots, la somme de deux vecteurs doit suivre ces 5 axiomes:

    \begin{center}
    \begin{minipage}[t]{0.45\textwidth}\vspace{0pt}
    Somme de deux vecteurs:
    \begin{enumerate}
        \item Stable.
        \item Commutative.
        \item Associative.
        \item Existence du vecteur nul.
        \item Existence de l'inverse.
    \end{enumerate}

    \end{minipage}
    \hfill
    \begin{minipage}[t]{0.45\textwidth}\vspace{0pt}
    Produit par un scalaire:
    \begin{enumerate}
        \item Stable.
        \item Distributive sur une somme de vecteur.
        \item Distributive sur une somme de scalaire.
        \item Associative.
        \item Existence de l'unité.
    \end{enumerate}

    \end{minipage}
    \end{center}
}

\parag{Exemple 1}{
    Si on a $V = \mathbb{R}^{n}$, avec l'addition et la multiplication par un scalaire habituelles, alors c'est bien un espace vectoriel.
}

\parag{Exemple 2}{
    Prenons $V = \mathbb{R}^{m \times n}$ avec l'addition et la multiplication par un scalaire habituelles pour des matrices.

    On pourrait démontrer que les axiomes tiennent pour cet ensemble et ces opérations, donc $\mathbb{R}^{m \times n}$ est un espace vectoriel, et les vecteurs sont les matrices.
}

\parag{Exemple 3}{
    Si on prend $V = \mathbb{P}_n$, l'ensemble des polynômes de degré $n$ ou moins, avec les opérations décrites ci-dessus.

    Un élément $p$ de $\mathbb{P}_n$ est un polynôme de degré au plus $n$, c'est-à-dire que $p$ est une fonction de $\mathbb{R}$ vers $\mathbb{R}$ telle que
    \[p\left(t\right) = a_0 + a_1 t + a_2 t^2 + \ldots + a_n t^n\]
    pour tout $t \in \mathbb{R}$ avec certains coefficients fixés $a_0, \ldots, a_n \in \mathbb{R}$.

    Alors on définit leur somme notée $p + q$ comme étant le polynôme tel que
    \[\left(p + q\right)\left(t\right) = p\left(t\right) + q\left(t\right) = a_0 + \ldots + a_n t^{n} + b_0 + \ldots + b_n t^n = \left(a_0 + b_0\right) + \ldots + \left(a_n + b_n\right)t^{n}\]

    Pour tout $t \in \mathbb{R}$, on remarque que $p + q$ est dans $\mathbb{P}_n$.

    Étant donné un scalaire $c \in \mathbb{R}$, on définit le produit de $c$ et $p$, noté $cp$, comme étant le polynôme tel que
    \[\left(cp\right)\left(t\right) = c p\left(t\right) = \left(ca_0\right) + \ldots \left(c a_n\right)t^n\]

    Pour tout $t \in \mathbb{R}$, on remarque que $cp$ est dans $\mathbb{P}_n$.

    Il est clair que la somme des polynômes est commutative et associatives.

    Soit $z$ le polynôme tel que $z\left(t\right) = 0$ pour tout $t \in \mathbb{R}$. Notons que $z$ est dans $\mathbb{P}_n$, et que $\left(p + z\right)\left(t\right) = p\left(t\right) + z\left(t\right) = p\left(z\right)$, donc $p + z = p$.

    On définit $-p = \left(-1\right)p$.

    On peut démontrer toutes les autres propriétés, donc $\mathbb{P}_n$ est un espace vectoriel.

}

\parag{Propriétés}{
    Soit $V$, un espace vectoriel. Alors, nous avons les propriétés suivantes:
    \begin{enumerate}
        \item Le vecteur nul $\bvec{0}$ est unique.
        \item Pour $\bvec{u} \in V$ donné, le vecteur $-\bvec{u}$ est unique.
        \item $0 \bvec{u} = 0$
        \item $c \bvec{0} = \bvec{0}$
        \item $-\bvec{u} = \left(-1\right)\bvec{u}$
    \end{enumerate}

    \subparag{Preuve}{
        Laissée au lecteur.
    }
}

\subsection{Sous-espaces vectoriels}
\parag{Définition de sous-espace vectoriel}{
    On appelle un \important{sous-espace vectoriel}, ou, en abrégé, sous-espace, d'un espace vectoriel $V$ toute sous-ensemble $H$ de $V$ possédant les trois propriétés suivantes:
    \begin{enumerate}
        \item Le vecteur nul de $V$ appartient à $H$.
        \item $H$ est stable par l'addition vectorielle, i.e:
        \[\bvec{u}, \bvec{v} \in H \implies \bvec{u} + \bvec{v} \in H\]
        \item $H$ est stable par la multiplication par un scalaire, i.e, pour tout scalaire $c$:
        \[\bvec{u} \in H \implies c \bvec{u} \in H\]

    \end{enumerate}

    \subparag{Remarque}{
        Il est presque suffisant d'avoir les propriétés (2) et (3) pour avoir un sous-espace, puisque $0\cdot \bvec{u} = \bvec{0}$. Cependant, cela pose problème si $H$ est complètement vide. Uniquement dans ce cas, les propriétés (2) et (3) tiennent, mais la propriété (1) ne tient pas.
    }

}

\parag{Exemple 1}{
    Soit $V = \mathbb{R}^{n}$ et
    \[H = \left\{\bvec{x} \in \mathbb{R}^{n} \telque x_1 + \ldots + x_n = 0\right\}\]

    On remarque que les trois propriétés tiennent:
    \begin{enumerate}
        \item $\bvec{0}$ est dans $H$. En effet: \[x_1 + \ldots + x_n = 0 + \ldots + 0 = 0\]
        \item Si $\bvec{u}, \bvec{v} \in H$, alors $\bvec{u} + \bvec{v} \in H$, en effet:
        \[u_1 + \ldots + u_n = 0 \text{ et } v_1 + \ldots + v_n = 0 \implies \left(u_1 + v_1\right) + \ldots + \left(u_n + v_n\right) = 0\]
        \item Si $\bvec{u} \in H$ et $c$ est un scalaire, alors $c \bvec{u} \in H$, en effet:
        \[u_1 + \ldots + u_n = 0 \implies c u_1 + \ldots + c u_n = c\left(u_1 + \ldots + u_n\right) = c\cdot0 = 0\]
    \end{enumerate}

    Donc, $H$ est un sous-espace vectoriel de $\mathbb{R}^{n}$.

}

\parag{Non-exemple 1}{
    Soit $V = \mathbb{R}^{n}$ et
    \[H = \left\{\bvec{x} \in \mathbb{R}^{n} \telque x_1 + \ldots + x_n = 1\right\}\]

    Ce n'est pas un sous-espace vectoriel de $\mathbb{R}^{n}$, puisqu'il n'a pas le vecteur nul.
}

\parag{Non-exemple 2}{
    Soit $V = \mathbb{R}^{n}$ et
    \[H = \left\{\bvec{x} \in \mathbb{R}^{n} \telque x_1 \geq 0, \ldots, x_n \geq 0\right\}\]

    Cependant, si on prend $\bvec{u} \in H$, $\bvec{u} \neq \bvec{0}$, et qu'on le multiplie par un scalaire négatif, alors il n'appartient pas à $H$. La troisième propriété ne tient donc pas.
}

\parag{Exemple 2}{
    Soit $V = \mathbb{R}^{n \times n}$ et
    \[H = \left\{A \in \mathbb{R}^{n \times n} \telque A = A^{T}\right\}\]

    On appelle de telles matrices, des matrices symétriques. Alors, on a:
    \begin{enumerate}
        \item $\bvec{0} \in H$ puisque la matrice nulle est égale à elle-même après avoir été transposée $(0^T = 0)$.
        \item Si $A,B \in \mathbb{R}^{n\times n}$ sont dans $H$, alors $C = A+B \in H$. En effet:
            \[\left(A + B\right)^{T} = A^{T} + B^{T} = A + B\]

        \item Si $A \in \mathbb{R}^{n\times n}$ est dans $H$ et $c \in \mathbb{R}$ est un scalaire, alors $B = cA$ est dans $H$. En effet:
            \[B^{T} = \left(cA\right)^{T} = cA^{T} = cA = B\]

    \end{enumerate}

    Donc, l'ensemble des matrices symétriques de taille $n \times n$ est un sous-espace vectoriel de $\mathbb{R}^{n\times n}$.
}

\parag{Exemple 3}{
    Soit $V = \mathbb{P}_5$, et $H = \mathbb{P}_3$. On a bien que $H$ est un sous-ensemble de $V$. De plus:
    \begin{enumerate}
        \item $z\left(t\right) = 0$ pour tout $t \in \mathbb{R}$ définit bien un polynôme nul, qui est dans $\mathbb{P}_5$ et dans $\mathbb{P}_3$.
        \item Pour tous $p, q \in \mathbb{P}_3$, on a bien que $p + q \in \mathbb{P}_3$ (on pourrait le démontrer plus formellement, mais on l'a déjà fait dans le cas général de $\mathbb{P}_n$).
        \item Pour tout $p \in \mathbb{P}_3$ et $c \in \mathbb{R}$, on a bien que $cp \in \mathbb{P}_3$.
    \end{enumerate}

Donc $\mathbb{P}_3$ est un sous-espace vectoriel de $\mathbb{P}_5$. Plus généralement, $\mathbb{P}_m$ est un sous-espace vectoriel de $\mathbb{P}_n$ si et seulement si $m \leq n$.

On aurait pu argumenter en disant que $\mathbb{P}_3$ est un sous-ensemble de $\mathbb{P}_5$, et il est un espace vectoriel quand on utilise les mêmes opérations, donc c'est un sous-espace.
}

\parag{Non-exemple 4}{
    $\mathbb{R}^{2}$ n'est pas un sous espace vectoriel de $\mathbb{R}^{3}$, puisque $\mathbb{R}^2$ n'est pas un sous-ensemble de $\mathbb{R}^3$.

    Cependant, $V = \left\{x \in \mathbb{R}^3 \telque x_3 = 0\right\}$ est un sous-espace vectoriel de $\mathbb{R}^3$.

    En d'autres mots, par exemple,
    \[\begin{bmatrix} 1 \\ 2 \\ 0 \end{bmatrix} \neq \begin{bmatrix} 1 \\ 2 \end{bmatrix} \]

}

\parag{Exemple 5}{
    Soit $V$ est un espace vectoriel et $H = \left\{\bvec{0}\right\} \subset V$ où $\bvec{0}$ est le vecteur nul de $V$. Alors, $H$ est un sous-espace vectoriel de $V$.

    On appelle $\left\{\bvec{0}\right\}$ le \important{sous-espace nul} ou \important{sous-espace trivial}.
}

\parag{Exemple 6 (sous-espace engendré)}{
    Soit $V$ un espace vectoriel. Soient $\bvec{v}_1, \ldots, \bvec{v}_3 \in V$ des vecteurs de $V$. L'ensemble
    \[H = \vect\left\{\bvec{v}_1, \bvec{v}_2, \bvec{v}_3\right\} = \left\{c_1 \bvec{v}_1 + c_2 \bvec{v}_2 + c\bvec{v}_3, \mathspace c_1, c_2, c_3 \in \mathbb{R}\right\}\]
    de toutes les combinaisons linéaires de ces trois vecteurs est un sous-ensemble de $V$.

    De plus, c'est un sous-espace vectoriel de $V$. En effet:
    \begin{enumerate}
        \item $\bvec{0} \in H$, puisque
        \[0 \bvec{v}_1 + 0 \bvec{v}_2 + 0 \bvec{v}_3 = \bvec{0}\]
        \item Si $\bvec{u} \in H$, alors il existe $c_1, c_2, c_3$ tels que
        \[\bvec{u} = c_1 \bvec{v}_1 + c_2 \bvec{v}_2 + c_3 \bvec{v}_3\]
        et de la même manière pour $\bvec{v} \in H$, avec des scalaires $d_1, d_2, d_3 \in \mathbb{R}$. Alors, on a:
        \[\bvec{u} + \bvec{v} = \left(c_1 + d_1\right) \bvec{v}_1 + \left(c_2 + d_2\right)\bvec{v}_2 + \left(c_3 + d_3\right) \bvec{v}_3 \in H\]

        \item La preuve est similaire, et elle est laissée en exercice au lecteur.
    \end{enumerate}

}

\end{document}
