\documentclass{article}

% Expanded on 2021-09-22 at 22:30:01.

\usepackage{../../style}

\title{Algèbre linéaire}
\author{Joachim Favre}
\date{Jeudi 23 septembre 2021}

\begin{document}
\maketitle

\lecture{1}{2021-09-23}{Début des systèmes d'équations linéaires}{
\begin{itemize}[left=0pt]
    \item Explication de l'organisation du cours.
    \item Définition du concept d'équation linéaire et de système.
    \item Définition des systèmes triangulaires.
    \item Explication des opérations élémentaires.
    \item Définition de la forme matricielle, ainsi que des matrices des coefficients et matrices augmentées.
\end{itemize}

}

\section{Organisation}
\parag{Moodle}{
    Toutes les informations sont sur Moodle, y compris les slides prises en ``live''. Il faut aller le consulter régulièrement.
}

\parag{Examen blanc}{
    En novembre, il y aura un examen blanc, pour qu'on puisse voir notre niveau.
}

\parag{Ouvrage de référence}{
    L'ouvrage de référence est ``Algèbre linéaire et applications'' écrit par David C. Lay (référence sur Moodle). Il donne des explications détaillées et des explication supplémentaire.
}


\section{Équations linéaires}
\subsection{Le cas d'une unique équation}
\parag{Définition des équations linéaires}{
    Pour des variables (= inconnues) $x_1, \ldots, x_n$, une \important{équation linéaire} est une équation de la forme
    \[a_1 x_1 + a_2 x_2 + \ldots + a_n x_n = b\]
    pour des nombres $a_1, a_2, \ldots, a_n$ et $b$ indépendants des variables.
}

\parag{Exemples}{
   Les équations suivantes sont des équations linéaires:
   \begin{itemize}
       \item $5x_1 + 3x_2 = 7$
       \item 0 = 5
       \item $\sqrt{7} = x_1 + x_2$
       \item $e^3 x_1 = 9 + x_2$
       \item $x_1 + x_2 + x_3 + x_4 + x_5 = 0$
       \item $5x_1 + x_{1000} = -\pi$
   \end{itemize}

   Les équations suivantes ne sont pas des équations linéaires:
   \begin{itemize}
       \item $e^{x_1} = e^{x_2}$
       \item $x_1 x_2 + x_1 = 0$
       \item $\sqrt{x_1 + x_2 + x_3} = \sqrt{9}$
   \end{itemize}

}

\parag{Exemple en deux dimensions}{
    On dessine les droites $x_1 = 2x_2 = -1$ et $-x_1 + 3x_2 = 3$ sur deux repères $x_1, x_2$.

    \imagehere{ExempleDroites.png}

    Gauche: $x_1 = 0$, donc $-2x_2 = -1 \implies x_2 = \frac{1}{2}$, donc $\left(0, \frac{1}{2}\right)$ est un point. De la même manière, en prenant $x_2 = 0$, on obtient $x_1 = -1$, donc $\left(-1, 0\right)$ est un autre point par lequel passe cette droite. On peut donc la dessiner.

    Droite: Si $x_1 = 0$, alors $x_2 = 1$, donc on a le point $\left(0, 1\right)$. De la même manière, en prenant $x_2 = 0$, on a $x_1 = -3$, donc le point $\left(-3, 0\right)$ fait aussi partie de la droite, qu'on peut dessiner.

    Ces droites sont donc l'ensemble des solutions à ces équations.
}


\subsection{Système (d'équations) linéaires}
\parag{Définition des systèmes d'équations linéaires, des solutions et des ensembles solutions}{
    Un \important{système d'équation linéaires} est une collection de une ou plusieurs équations linéaire en des variables $x_1, \ldots, x_n$. S'il y a $m$ équations en $n$ variables, on écrit
    \begin{systemofequations}{}
    &\ a_{11} x_1 + \ldots + a_{1n} x_n = b_1 \\
    &\ \vdots \\
    &\ a_{m1} x_1 + \ldots + a_{mn} x_n = b_m
    \end{systemofequations}
    avec $a_{ij}$, le coefficient de $x_j$ dans la ième équation.

    Étant donnée une liste de nombres $\left(s_1, \ldots, s_n\right)$, on dit que cette liste est une \important{solution} (du système) si les $m$ équations sont vraies quand on remplace $x_1$ par $s_1$, \ldots, et $x_n$ par $s_n$.

   L'ensemble des solutions d'un système est son \important{ensemble solution.} Deux systèmes sont \important{équivalents} s'ils ont le même ensemble solutions. }

\parag{Exemples}{
    Les droites se croisent en un point. Un point d'une droite satisfait son équation, donc si un point appartient aux deux droites simultanément, il est une solution au système d'équations.

    \imagehere{ExemplesNSolutions.png}

    Dans le cas où les droites se croisent, il y a une solution unique. Si les deux droites sont parallèles, il n'existe aucun point qui appartient simultanément aux deux droites, donc l'ensemble de solution est vide, noté $\o$. Dans le cas où les droites sont les mêmes, il y a une infinité de solution.
}

\parag{Trois variables}{
    Une équation avec $a_1 x_1 + a_2 x_2 + a_3 x_3 = b$ définit un plan. Le croisement entre deux plan définit une droite.

    Du coup, un système linéaire avec $m$ équations en $n = 3$ variables définit $m$ plans dans l'espace en 3D.

    \imagehere{ExemplePlans.png}

    Sur le premier des deux dessins, le plan bleu est parallèle à la droite dessinée par les plans verts et rouges, donc il n'y a pas de solution.
}

\parag{Théorème}{
    Un système linéaire a zéro (aucune solution n'existe), une (la solution existe et est unique) ou une infinité (des solutions existent, mais elles ne sont pas uniques) de solutions.

    \subparag{Preuve}{
        \begin{itemize}[left=0pt]
            \item Zéro solution:
            \item Une solution:
            \item Plus qu'une solution (on complètera plus tard, mais on peut la faire; c'est une preuve par l'absurde. Si on a deux solutions, il y a une droite qui relie ces deux points dans l'espace en n-D et on peut montrer que, si on a ces deux solutions, alors tous les points de la droite qui les relient sont aussi des solutions, ce qui est une contradiction).
        \end{itemize}
        \qed
    }
}

\subsection{Système triangulaire}
\parag{Trouver toutes les solutions}{
    On a un système linéaire de $m \times n$ (avec $m$, le nombre d'équation, et $n$ le nombre de variables) avec coefficients $a_{ij}$ (avec $i \in \left\{1, \ldots, m\right\}$ et $j \in \left\{1, \ldots, n\right\}$):
    \begin{systemofequations}{}
    &\ a_{11} x_1 + \ldots + a_{1n} x_n = b_1 \\
    &\ \vdots \\
    &\ a_{m1} x_1 + \ldots + a_{mn} x_n = b_m
    \end{systemofequations}

    Certains systèmes sont plus faciles à résoudre que d'autres. Les diagonaux sont plus simples que le triangulaires, qui sont plus simples que les ``généraux''.  La stratégie va être de transformer un système général en système triangulaire (ou aussi triangulaire que possible).
}

\parag{Exemple triangulaire}{
   \begin{systemofequations}{}
   &\ 2x_1 + 4x_2 + 2x_3 = 1 \\
   &\ 2x_2 + 2x_3 = 3 \\
   &\ 4x_3 = 2
   \end{systemofequations}

   \begin{itemize}[left=0pt]
       \item $4x_3 = 2 \implies x_3 = \frac{1}{2}$
       \item $2x_2 + 2x_3 = 3 \implies 2x_2 = 3 - 2x_3 = 2 \implies x_2 = 1$
       \item $2x_1 + 4x_2 + 2x_3 = 1 \implies 2x_1 = 1 - 4x_2 - 2x_3 = 1 - 4 - 1 = -4 \implies x_1 = -2$
   \end{itemize}

   Le système a donc une solution unique: $\left(-2, 1, \frac{1}{2}\right)$. On peut vérifier la solution.
}

\parag{Systèmes triangulaires}{
    Plus généralement, un système est triangulaire, s'il est de la forme:
    \begin{systemofequations}{}
    &\ a_{11} x_1 + a_{12} x_2 + \ldots + a_{1n} x_n = b_1 \\
    &\ a_{22} x_2 + \ldots + a_{2n} x_n = b_2  \\
    &\ \cdots
    \end{systemofequations}

    On peut avoir $m \neq n$, naturellement. Si le système n'est pas triangulaire, on va modifier le système pour le rendre triangulaire ou ``presque'' triangulaire, sans changer l'ensemble des solution on ne veut ni ``perdre'' ni ``créer'' de solutions.
}

\subsection{Opérations élémentaires}
\parag{Théorème des opérations élémentaires}{
    Les opérations suivantes ne changent pas l'ensemble solution:
    \begin{enumerate}
        \item Permuter deux équations.
        \item Multiplier une équation par un nombre non nul.
        \item Additionner à une équation un multiple d'une autre équation.
    \end{enumerate}

    On les appelle des \important{opérations élémentaires}.

    \subparag{Preuve}{
        Laissée au lecteur.
    }

    \subparag{Remarque}{
        Toutes les opérations sont reversible. Cela peut être démontré facilement.
    }
}

\parag{Exemple}{
    On a
    \begin{systemofequations}{}
    &\ 2x_1 - 3x_2 +5x_3 = 8  \\
    &\ x_1 + x_2 - 3x_3 = -7  \\
    &\ 4x_1 + 2x_2 + 2x_3 = 0
    \end{systemofequations}

    En permutant les équations $(a)$ et $(b)$, i.e. $\left(a\right) \leftrightarrow \left(b\right)$:
    \begin{systemofequations}{}
    &\ x_1 + x_2 - 3x_3 = -7  \\
    &\ 2x_1 - 3x_2 + 5x_3 = 8  \\
    &\ 4x_1 + 2x_2 + 2x_3 = 0
    \end{systemofequations}

    On va essayer de rentre les coefficients tout à gauche nul, la seule opération qui pourrait fonctionner est d'ajouter un multiple d'une autre équation. On va remplacer l'équation $(c)$ par l'équation $(c)$ moins quatre fois l'équation $(a)$ (les noms $\left(a\right), \left(b\right), \left(c\right)$ changent à chaque étape) et l'équation deux de manière similaire, i.e. $\left(c\right) \leftarrow \left(c\right) - 4\left(a\right)$ et $\left(b\right) \leftarrow \left(b\right) - 2\left(a\right)$:
    \begin{systemofequations}{}
    &\ x_1 + x_2 - 3x_3 = -7  \\
    &\ -5x_2 + 11x_3 = 22  \\
    &\ -2x_2 + 14x_3 = 28
    \end{systemofequations}

    On veut rendre le coefficient tout à gauche de la troisième équation nul. Cela semble compliqué en utilisant la première équation (le coefficient devant $x_1$ ne serait plus nul). On prend $\left(c\right) \leftarrow \frac{5}{2}\left(c\right)$:
    \begin{systemofequations}{}
    &\ x_1 + x_2 - 3x_3 = -7  \\
    &\ -5x_2 + 11x_3 = 22  \\
    &\ -5x_2 + 35x_3 = 70
    \end{systemofequations}

    On prend $\left(c\right) \leftarrow \left(c\right) - \left(b\right)$:
    \begin{systemofequations}{}
    &\ x_1 + x_2 - 3x_3 = -7  \\
    &\ -5x_2 + 11x_3 = 22  \\
    &\ 24x_3 = 48
    \end{systemofequations}

    Maintenant qu'on a rendu notre système triangulaire, il est très facile de trouver les solutions du système.
}

\parag{Pour aller plus loin}{
    Nous avons encore plusieurs choses à faire pour aller plus loin:
    \begin{enumerate}
        \item On peut adopter une \important{notation} plus efficace, qui se révèlera être bien plus qu'une notation (les matrices \smiley)
        \item On peut \important{simplifier davantage} (en transformant la matrice en ``diagonale'' et non en ``triangulaire'')
        \item Il n'est pas toujours possible d'obtenir un triangle parfait.
    \end{enumerate}
}

\subsection{Matrices}
\parag{Notation matricielle}{
    Les équations sont de la forme $a_1 x_1 + \ldots a_n x_n = b_n$. Vu qu'elles ont toutes la même forme, seuls les coefficients $a_1, \ldots, a_n, b_n$ sont importants. Si on a un système
    \begin{systemofequations}{}
    &\ a_{11} x_1 + \ldots + a_{1n} x_n = b_1 \\
    &\ \vdots \\
    &\ a_{m1} x_1 + \ldots + a_{mn} x_n = b_m
    \end{systemofequations}

    On collecte les nombres utiles dans deux tableaux appelés matrices:
\[\begin{bmatrix} a_{1,1} & \ldots & a_{1n} \\ \vdots &  & \vdots \\ a_{m1} & \ldots & \cdot a_{mn} \end{bmatrix} \mathspace \begin{bmatrix} a_{1,1} & \cdots & a_{1n} & b_1 \\ \vdots &   & \vdots & \vdots \\ a_{m1} & \ldots & a_{mn} & b_m \end{bmatrix}\]

    Sur la gauche, la \important{matrice des coefficients} $a_{ij}$ est de taille $m \times n$. Sur la droite, c'est la \important{matrice augmentée}, qui est de taille $m \times \left(n + 1\right)$ (on peut mettre des traitillés juste avant la dernière colonne de cette matrice augmentée, mais on n'est pas obligé).

    Chaque système linéaire correspond à une matrice augmentée, et vice versa.
}

\parag{Les opérations élémentaires}{
    Nous retrouvons ces mêmes opérations élémentaires pour une matrice augmenté:
    \begin{enumerate}
        \item Permuter deux équations. / Permuter deux lignes.
        \item Multiplier une équation par un nombre non nul. / Multiplier une ligne par un nombre non nul.
        \item Additionner à une équation un multiple d'une autre équation. / Additionner à une ligne un multiple d'une autre ligne.
    \end{enumerate}
}

\parag{Exemple}{
    On peut transformer notre système vu plus tôt en matrice augmentée:
    \[\begin{bmatrix} 2 & -3 & 5 & 8 \\ 1 & 1 & -3 & -7 \\ 4 & 2 & 2 & 0 \end{bmatrix} \]

    On peut permuter les deux premières lignes:
    \[\begin{bmatrix} 1 & 1 & -3 & -7 \\ 2 & -3 & 5 & 8 \\ 4 & 2 & 2 & 0 \end{bmatrix} \]

    On peut soustraire quatre fois la première ligne à la dernière ligne, et on peut soustraire deux fois la première à la seconde ligne:
\[\begin{bmatrix} 1 & 1 & -3 & -7 \\ 0 & -5 & 11 & 22 \\ 0 & -2 & 14 & 28 \end{bmatrix} \]

    On passe les quelques étapes qu'on a vue plus haut, puis on arrive à
\[\begin{bmatrix} 1 & 1 & -3 & -7 \\ 0 & -5 & 11 & 22 \\ 0 & 0 & 24 & 48 \end{bmatrix}\]
    qui est bien une matrice triangulaire, et qui est la matrice augmentée du système triangulaire qu'on avait trouvé plus tôt. On peut retourner au monde des systèmes pour résoudre nos équations.
}


\end{document}
